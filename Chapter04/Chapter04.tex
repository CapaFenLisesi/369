\chapter{Heat and Work}\label{s4}

\section{Brief History of Heat and Work}\label{s4.1}
In 1789 the French scientist Antoine Lavoisier published a famous treatise
 on Chemistry
which, amongst other things, demolished the then 
prevalent theory of combustion. 
This theory, known to history as the {\em phlogiston theory}, is so
extraordinary stupid that it is not even worth describing.
In place of  phlogiston theory, Lavoisier proposed the first reasonably 
sensible scientific interpretation  of heat. Lavoisier pictured heat 
as an invisible, tasteless, odourless, weightless 
 fluid, which he called {\em calorific fluid}. He postulated that 
hot bodies 
contain more of this  fluid than cold bodies. Furthermore, he
suggested  that the constituent particles 
of calorific  fluid repel one another, causing  heat to flow from
hot to cold bodies when they are placed in thermal contact. 

The modern interpretation of heat is, or course,  somewhat different to Lavoisier's
 {\em calorific theory}. Nevertheless, there is an important subset
of problems involving heat flow for which Lavoisier's approach is
rather useful. 
These problems often crop up as examination questions.
 For example: ``A clean dry copper
calorimeter contains 100 grams of water at 30$^\circ$ degrees centigrade.
A 10 gram block of copper heated to 60$^\circ$ centigrade is added. What is the final
temperature of the mixture?''. 
How do we approach this type of problem? Well, 
according to Lavoisier's theory, there is an 
analogy between heat flow and incompressible 
fluid  flow  under gravity. 
 The same volume of liquid added to 
containers of different cross-sectional area  fills them  to different heights.
If the volume is $V$,  and the cross-sectional area is $A$,  then the height is
$h = V/A$. In a similar manner, the same quantity of heat added to different bodies
causes them to rise to different temperatures. If $Q$ is the heat and
$\theta$ is the (absolute) temperature then 
$\theta = Q/C$, where the constant $C$ is termed the {\em heat capacity}.
[This is a somewhat oversimplified example. In general, the heat capacity is
a function of temperature, so that $C=C(\theta)$.]
 Now, if two containers filled to different heights 
with a free flowing incompressible fluid
are connected together at the bottom, via a small pipe, 
then  fluid will  flow under gravity,
from one
to the other, until the two heights are the same. The final height is easily 
calculated by equating the total fluid volume in the initial and final
states. Thus,
\begin{equation}
h_1 \,A_1 + h_2 \,A_2 = h \,A_1 + h \,A_2,
\end{equation}
giving
\begin{equation}
h = \frac{h_1 \,A_1 + h_2\, A_2}{A_1 + A_2}.
\end{equation}
Here, $h_1$ and $h_2$ are the initial heights in the two containers, $A_1$ and
$A_2$ are the corresponding cross-sectional areas, and $h$ is the final height.
Likewise, if two bodies, initially at different temperatures, are brought into
thermal contact then heat will flow, from one to the other, until the two
temperatures are the same. The final temperature is calculated by equating the total
heat in the initial and final states. Thus,
\begin{equation}
\theta_1\, C_1 + \theta_2\, C_2 = \theta\, C_1 + \theta\, C_2,
\end{equation}
giving
\begin{equation}
\theta = \frac{\theta_1 \,C_1 + \theta_2\, C_2}{C_1 + C_2},
\end{equation}
where the meaning of the various symbols should be self-evident.

The analogy between heat flow and fluid flow works because in Lavoisier's theory
heat is a {\em conserved}\/ quantity, just 
like the volume of an incompressible fluid. In fact,
Lavoisier postulated that heat was an element. Note that 
atoms were thought to be indestructible 
before
nuclear reactions were discovered, so the 
total amount of each element in the Universe was assumed to be a constant.
If Lavoisier had cared to formulate a law of thermodynamics from his calorific
theory then
he would have said that the total amount of heat in the Universe was a constant.
 

In 1798 Benjamin Thompson, an Englishman who spent his early years in 
pre-revolutionary
America, was  minister for war and police in the German state of Bavaria.
One of his jobs was to oversee the boring of cannons in the state arsenal. 
Thompson was struck by the
enormous, and seemingly inexhaustible, amount of heat generated in this process.
 He simply could not  understand where all this
heat was coming from. According to Lavoisier's calorific theory,  the heat 
must flow into the cannon from its immediate surroundings, which should, therefore,
 become colder.
The flow should  also  eventually cease when all of the available heat has been
extracted.
In fact, Thompson observed that the surroundings of the cannon
got hotter, not colder,
and that the heating process  continued unabated  as long as the
boring machine was operating. Thompson postulated that some of the mechanical
work done
on the cannon by the boring machine was being converted into heat. At the time,
this was quite a revolutionary concept, and most people were not ready to accept it.
This is somewhat surprising, since by the end of the eighteenth century
the conversion of heat into
work, by steam engines, was quite commonplace.
Nevertheless, the conversion of work into heat did not gain broad acceptance until
1849, when an English physicist called
 James Prescott Joule published the results of a long and
painstaking series of experiments. Joule confirmed that work could  indeed 
be converted
into heat. Moreover, he found that the same amount of work always generates
 the same quantity of
heat. This is
 true regardless of the nature of the work ({\em e.g.}, mechanical, electrical,
{\em etc}.). Joule was able to formulate what became known as
the {\em work equivalent of heat}. 
Namely, that 1 newton meter of work is equivalent to $0.241$ calories of heat.
A calorie is the amount of heat required to raise the temperature of 1 gram of
water by 1 degree centigrade. Nowadays, we measure both heat and work in the
same units, so that one newton meter, or joule, of work is equivalent to
one joule of heat.

In 1850, the German physicist Clausius correctly
postulated that the essential conserved quantity
is neither heat nor work, but some combination of the two which quickly became
known as {\em energy}, from the Greek {\em energia}\/ meaning ``in work.''\@
According to Clausius, the change in the internal energy of a macroscopic body
can be written
\begin{equation}
{\mit\Delta} E = Q - W,
\end{equation}
where $Q$ is the heat {\em absorbed}\/ from the surroundings, and $W$ is the work done 
 {\em on}\/ the surroundings. This relation is known as {\em the first law of 
thermodynamics}. 

\section{Macrostates and Microstates}
In describing a system made up of a great many particles,
it is usually  possible to specify
some macroscopically measurable independent parameters $x_1$, $x_2, \cdots, x_n$
which affect the  particles' equations of motion. 
These parameters are termed the {\em external parameters}\/ of the system.
Examples of such parameters are the volume (this gets into the equations
of motion because the potential energy  becomes infinite when a particle
strays outside the available volume) and any applied electric and magnetic fields.
A {\em microstate}\/ of the system is defined as a state for
which the motions of the individual
particles are completely specified (subject, of course, 
 to the unavoidable limitations imposed
by the uncertainty principle of quantum mechanics).
In general, the overall energy of a given microstate $r$  is a function
of the external parameters:
\begin{equation}
E_r \equiv E_r(x_1, x_2, \cdots, x_n).\label{e4.6}
\end{equation}
A {\em macrostate}\/ of the system is defined by specifying the external parameters,
and any other constraints to which the system is subject. For example, if we
are dealing with an isolated system ({\em i.e.}, one that can neither exchange heat with nor
do work on its surroundings) then the macrostate might be specified by giving the
values of the volume and the constant total energy. 
For a many-particle system, there are generally a very great
number of microstates which are consistent with a given macrostate.

\section{Microscopic Interpretation of Heat and Work}
Consider a macroscopic system $A$ which is 
known to be in a given macrostate. To be more exact, 
consider an ensemble
of similar macroscopic systems $A$, where  each system in the ensemble
is in one of the many microstates consistent with the given
macrostate. There are two fundamentally different ways
in which the average energy of $A$ can change due to interaction with its 
surroundings. If the external parameters of the system remain constant then the 
interaction is termed a purely {\em thermal}\/ interaction. Any change in the average
energy of the system is attributed to an exchange of {\em heat}\/ with its environment.
Thus,
\begin{equation}
{\mit\Delta} \bar{E} = Q,
\end{equation}
where $Q$ is the heat absorbed by the system. On a microscopic level, the
energies of the individual microstates are  unaffected by the absorption of heat.
In fact, it is the {\em distribution}\/ of the systems in the ensemble 
over the various microstates which is modified.

Suppose that the system $A$ is thermally insulated from its environment. This can be
achieved by surrounding it by an {\em adiabatic}\/ envelope ({\em i.e.}, an envelope 
fabricated out of a material which is a poor conductor of heat, such a fiber glass).
Incidentally, the term adiabatic is derived from the Greek {\em adiabatos}\/ which
means ``impassable.''\@ In scientific terminology, an adiabatic process is one in
which there is no exchange of heat. The system $A$ is still capable of interacting
with its environment via its external parameters. This type of interaction is
termed {\em mechanical}\/ interaction, and any change in the average energy of the
system is attributed to work done on it by its surroundings. Thus,
\begin{equation}
{\mit\Delta} \bar{E} = -W,
\end{equation}
where $W$ is the work done by the system on its environment. On a microscopic level,
the energy of the 
system changes because the energies of the individual microstates are
functions of the external parameters [see Eq.~(\ref{e4.6})]. 
Thus, if the external parameters
are changed then, in general, the energies of all of the systems in the ensemble
are modified (since each is in a specific microstate). Such a modification 
 usually gives rise to a redistribution of the systems in the ensemble over the
accessible microstates  (without any heat
exchange with the environment). Clearly, from a microscopic viewpoint, performing
 work on
a macroscopic system is quite a complicated process. Nevertheless, macroscopic work
is a quantity which can be readily measured experimentally. For instance, if
the system $A$ exerts a force ${\bf F}$ on its immediate surroundings, and 
the change in external parameters corresponds to a displacement ${\bf x}$ 
of the center of mass of the system, then the work done {\em by}
$A$ on its surroundings
is simply
\begin{equation}
W =  {\bf F}\!\cdot \!{\bf x}
\end{equation}
{\em i.e.}, the product of the force and the displacement along the line of action
of the force.
In a general interaction of the system $A$ with its environment there is both
heat exchange and work performed. We can write
\begin{equation}
Q \equiv {\mit\Delta} \bar{E} + W,\label{e4.10}
\end{equation}
which serves as the general definition of the absorbed heat $Q$ (hence, the
equivalence sign). The quantity $Q$ is
simply the change in the mean energy of the system
which is not due to the modification of the
external parameters. Note that the notion of a quantity of heat has no independent
meaning apart from Eq.~(\ref{e4.10}). The mean energy $\bar{E}$ and work 
performed $W$ are
both physical quantities which can be determined experimentally, whereas 
$Q$ is merely a derived quantity.

\section{Quasi-Static Processes}\label{s4.4}
Consider the special case of an interaction of the system $A$ with its surroundings
which is carried out so slowly that $A$ remains arbitrarily close to equilibrium
at all times. Such a process is said to be {\em quasi-static}\/ for
the system $A$. In practice, a quasi-static process must be carried out on
a time-scale which is much longer than the relaxation time of the system. 
Recall
that the relaxation time is the typical time-scale for the  system to return
to equilibrium after being suddenly disturbed (see Sect.~\ref{s3.10}).

A finite quasi-static change can be built up out of many infinitesimal changes.
The infinitesimal heat  $\,\dbar Q$ absorbed by the system when infinitesimal work
$\,\dbar W$ is done on its environment 
and its average energy changes by $d \bar{E}$ is
given by
\begin{equation}
\dbar Q \equiv d\bar{E} + \dbar W.\label{e4.11}
\end{equation}
The special symbols $\,\dbar W$ and $\,\dbar Q$ are introduced to emphasize that the
work done and heat absorbed are infinitesimal quantities which
do {\em not}\/ correspond to the 
difference between two works or two heats. Instead, the work done and heat absorbed
depend on  the interaction {\em process}\/ itself. Thus, it makes no sense to talk
about the work in the system before and after the process, or the difference between
these.

If the external parameters of the system have the values $x_1$, $\cdots, x_n$ then
the energy of the system in a definite microstate $r$ can be written
\begin{equation}
E_r = E_r(x_1,\cdots, x_n).
\end{equation}
Hence, if the external parameters are changed by infinitesimal amounts, so that
$x_\alpha \rightarrow x_\alpha + dx_\alpha$ for $\alpha$ in  the range 1 to $n$, then
the corresponding change in the energy of the microstate is 
\begin{equation}
d E_r = \sum_{\alpha =1}^n \frac{\partial E_r}{\partial x_\alpha}\,dx_\alpha.
\end{equation}
The work $\,\dbar W$ done {\em by}\/ the system when it remains in this particular
state $r$ is 
\begin{equation}
\dbar W_r = - d E_r = \sum_{\alpha=1}^n X_{\alpha\,r}\,dx_\alpha,
\end{equation}
where 
\begin{equation}
X_{\alpha\,r} \equiv -\frac{\partial E_r}{\partial x_\alpha}
\end{equation}
is termed the {\em generalized force}\/ (conjugate to the external parameter $x_\alpha$)
in the state $r$. Note that if $x_\alpha$ is a displacement then $X_{\alpha\,r}$ is
an ordinary force.

Consider now an ensemble of systems. Provided that the external
parameters of the system are changed quasi-statically, the generalized forces
$X_{\alpha\,r}$ have well defined mean values which are calculable from the 
distribution of systems in the ensemble characteristic of the instantaneous 
macrostate. 
The macroscopic work $\,\dbar W$ resulting
from an infinitesimal quasi-static change of the external parameters is obtained by
calculating the decrease in the mean energy resulting from the parameter change.
Thus,
\begin{equation}
\dbar W = \sum_{\alpha =1}^n \bar{X}_\alpha\, dx_\alpha,\label{e4.16}
\end{equation}
where
\begin{equation}
\bar{X}_\alpha \equiv -\overline{ \frac{\partial E_r}{\partial x_\alpha}}\label{e4.17}
\end{equation}
is the {\em mean}\/ generalized force conjugate to $x_\alpha$. The mean value is
calculated from the equilibrium distribution of systems in the ensemble corresponding to the external parameter values $x_\alpha$. The macroscopic work $W$ resulting
from a finite quasi-static change of external parameters can be obtained by
integrating Eq.~(\ref{e4.16}).

The most well-known example of quasi-static work in thermodynamics is that done by
pressure when the volume  changes. For simplicity, suppose that the
volume $V$ is the only external parameter of any consequence. The work done in 
changing the volume from $V$ to $V + dV$ is simply the product of the force and
the displacement (along the line of action of the force). By definition, the mean
equilibrium pressure $\bar{p}$ 
 of a given macrostate is equal to the normal force per unit area acting
on any surface element. Thus, the normal force acting on a surface element 
$d{\bf S}_i$ is $\bar{p}\,\,d{\bf S}_i$. Suppose that the surface element is subject
 to a
displacement $d{\bf x}_i$. The work done by the element is $\bar{p}\,\,d{\bf S}_i
\!\cdot\! d{\bf x}_i$. 
The total work done by the system is obtained by summing over all of the surface 
elements. Thus,
\begin{equation}
\dbar W = \bar{p}\, \,dV,
\end{equation}
where
\begin{equation}
dV = \sum_i d{\bf S}_i\!\cdot\! d{\bf x}_i
\end{equation}
is the infinitesimal volume change due to the displacement of the surface.
It follows from (\ref{e4.17}) that
\begin{equation}
\bar{p} = - \frac{ \partial\bar{E} }{\partial V},
\end{equation}
so the mean pressure is the generalized force conjugate to the volume $V$.

Suppose that a quasi-static process is carried out in which the volume is changed
from $V_i$ to $V_f$. In general, the mean pressure is a function of the volume, so
 $\bar{p} = \bar{p}(V)$. It follows that the macroscopic work done by the
system is given by
\begin{equation}
W_{if} = \int_{V_i}^{V_f} \dbar W = \int_{V_i}^{V_f} \bar{p}(V)\, dV.
\end{equation}
This quantity is just the ``area under the curve'' in a plot of $\bar{p}(V)$ versus
$V$.

\section{Exact and Inexact Differentials}\label{s4.5}
In our investigation of heat and work we have come across 
various infinitesimal objects such as $d\bar{E}$ and $\,\dbar W$.
It is instructive to examine these infinitesimals more closely.

Consider the purely mathematical problem where $F(x, y)$ is some general
function of two
independent variables $x$ and $y$. Consider the change in $F$ in
going from the point $(x$, $y)$ in the $x$-$y$ plane to the neighbouring point
($x+dx$, $y+dy$). This is given by
\begin{equation}
dF = F(x + dx, y+dy)- F(x,y),
\end{equation}
which can also be written
\begin{equation}
dF = X(x,y) \,dx + Y(x,y)\, dy,\label{e4.23}
\end{equation}
where $X= \partial F/\partial x$ and $Y=\partial F/\partial y$. Clearly, $dF$ is
simply the infinitesimal difference between two adjacent values of the function $F$.
This type of infinitesimal quantity is termed an {\em exact differential}\/ to
distinguish it from another type to be discussed presently.
If we move in the $x$-$y$ plane from an initial point $i\equiv (x_i$, $y_i)$ 
to a final point  $f \equiv (x_f$, $y_f)$ then the corresponding change in $F$ is
given by
\begin{equation}
{\mit\Delta} F = F_f - F_i = \int_i^f dF= \int_i^f(X\,dx + Y\,dy).
\end{equation}
Note that since the difference on the left-hand side depends only on the initial
and final points, the integral on the right-hand side can only depend on 
these points as well.
In other words, the value of the
integral is independent of the path taken in going from the initial to the final
point. This is the distinguishing feature of an exact differential. 
Consider an integral taken around a closed circuit in the $x$-$y$ plane. In this
case, the initial and final points correspond to the same point, so the difference
$F_f- F_i$ is clearly zero. It follows that the integral of an exact differential
over a closed circuit is always zero:
\begin{equation}
\oint dF \equiv 0.
\end{equation}

Of course, not every infinitesimal quantity is an exact differential. Consider
the infinitesimal object
\begin{equation}
\dbar G \equiv X'(x,y)\, dx + Y'(x,y)\, dz,\label{e4.26}
\end{equation}
where $X'$ and $Y'$ are two
general functions of $x$ and $y$. It is easy to test whether or not
an infinitesimal quantity is an exact differential. Consider the expression 
(\ref{e4.23}).
It is clear that since $X= \partial F/\partial x$ and $Y=\partial F/\partial y$
then
\begin{equation}
\frac{\partial X}{\partial y} = \frac{\partial Y}{\partial x}= \frac{\partial^2 F}
{\partial x \partial y}.
\end{equation}
Thus,
if
\begin{equation}
\frac{\partial X'}{\partial y} \neq \frac{\partial Y'}{\partial x}
\end{equation}
(as is assumed to be the case), then $\,\dbar G$ cannot be an exact differential, and
is instead termed an {\em inexact differential}. 
The special symbol $\,\dbar$ is used to denote an inexact differential.
Consider the integral of
$\,\dbar G$ over some path in the $x$-$y$ plane. In general, it is not true
that
\begin{equation}
\int_i^f \dbar G = \int_i^f (X'\, dx + Y' \,dy)
\end{equation}
is independent of the  path taken between the initial and final points.
This is the distinguishing feature of an inexact differential. In particular,
the integral of an inexact differential around a closed circuit is not necessarily
zero, so
\begin{equation}
\oint \dbar G \neq 0.
\end{equation}

Consider, for the moment, the solution of 
\begin{equation}
\dbar G = 0,
\end{equation}
which reduces to the ordinary differential equation
\begin{equation}
\frac{dy}{dx} = -\frac{X'}{Y'}.\label{e4.32}
\end{equation}
Since the right-hand side is a known function of $x$ and $y$, the above
equation defines
a definite direction ({\em i.e.}, gradient) at each point in the $x$-$y$ plane. The
solution  simply consists of drawing a system of curves in the
$x$-$y$ plane such that at any point the tangent to the curve is as specified
in Eq.~(\ref{e4.32}). This defines a set of curves which can be written
$\sigma (x, y) = c$, where $c$ is a labeling parameter.
It follows that
\begin{equation}
\frac{d\sigma}{d x}\equiv \frac{\partial \sigma}{\partial x}
+\frac{\partial\sigma}{\partial y}\frac{dy}{dx} = 0.\label{e4.33}
\end{equation}
The elimination of $dy/dx$ between Eqs.~(\ref{e4.32}) and (\ref{e4.33}) yields
\begin{equation}
Y'\, \frac{\partial \sigma}{\partial x} = X'\, \frac{\partial\sigma}{\partial y}
=\frac{X' \,Y'}{\tau},
\end{equation}
where $\tau(x, y)$ is function of $x$ and $y$. The above equation could equally
well be written
\begin{equation}
X' = \tau\, \frac{\partial \sigma}{\partial x},~~~~~Y' = \tau\,
 \frac{\partial \sigma}
{\partial y}.\label{e4.35}
\end{equation}
Inserting Eq.~(\ref{e4.35}) into Eq.~(\ref{e4.26}) gives
\begin{equation}
\dbar G = \tau \left( \frac{\partial \sigma}{\partial x}\,dx + \frac{\partial\sigma}
{\partial y}\,dy\right)= \tau\,d\sigma,
\end{equation}
or
\begin{equation}
\frac{\dbar G}{\tau} = d\sigma.
\end{equation}
Thus, dividing the inexact differential $\,\dbar G$ by $\tau$ yields the exact
differential $d\sigma$. A factor $\tau$ which possesses this property is
termed an {\em integrating factor}. Since the above analysis is quite general,
it is clear that an inexact differential involving {\em two}\/ independent variables always
admits of an integrating factor. Note, however, 
this is not generally the case for inexact
differentials involving  more than two variables.

After this mathematical excursion, let us return to physical situation of interest.
The macrostate of a macroscopic system can be specified by the values of the
external parameters ({\em e.g.}, the volume) and the mean energy $\bar{E}$. This, in
turn, fixes other parameters such as the mean pressure $\bar{p}$. Alternatively,
we can specify the external parameters and the mean pressure, which fixes the 
mean energy. Quantities such as $d\bar{p}$ and $d\bar{E}$ are infinitesimal
differences between well-defined quantities: {\em i.e.},
 they are exact differentials.
For example, $d\bar{E} = \bar{E}_f - \bar{E}_i$ is just the difference between the
mean energy of the system in the final macrostate $f$ and  the
initial macrostate $i$, in the limit where these two states are nearly the same.
It follows that if the system is taken from an initial macrostate $i$ to any
final macrostate $f$ the mean energy change is given by 
\begin{equation}
{\mit\Delta} \bar{E} = \bar{E}_f - \bar{E}_i = \int_{i}^f d\bar{E}.
\end{equation}
However, since the mean energy is just a function of the macrostate under consideration, $\bar{E}_f$ and $\bar{E}_i$ depend only on the  initial and final
states, respectively.
 Thus, the integral $\int d\bar{E}$ depends only on the initial and final
states, and not on the particular process  used to get between them.

Consider, now, the infinitesimal work done by the system in going from some
initial macrostate $i$ to some neighbouring final macrostate $f$. In general, 
$\,\dbar W = \sum \bar{X}_\alpha\,dx_\alpha$ is {\em not}\/ the difference between two
numbers referring to the properties of two neighbouring macrostates. Instead,
it is merely an infinitesimal quantity characteristic of the process of going
from state $i$ to state $f$. In other words, the work $\,\dbar W$ is in general
an inexact differential. The total work done by the system in going from any
macrostate $i$ to some other macrostate $f$ can be written as
\begin{equation}
W_{if} = \int_i^f \dbar W,
\end{equation}
where the integral represents the sum of the infinitesimal amounts of work
$\,\dbar W$ performed at each stage of the process. In general, the value of
the integral {\em does}
 depend on the particular process used in going from macrostate $i$
to macrostate $f$. 

Recall that in going from macrostate $i$ to macrostate $f$ the change ${\mit\Delta} \bar{E}$ 
{\em does not}
depend on the process used whereas the work $W$, in general, {\em does}. 
Thus, it follows from
the first law of thermodynamics, Eq.~(\ref{e4.10}),
 that the heat $Q$, in general, also depends on the 
process used. It follows that
\begin{equation}
\dbar Q \equiv  d \bar{E} + \dbar W
\end{equation}
is an inexact differential. However, by analogy with the mathematical example
discussed previously, there must exist some integrating factor, $T$, say, which
converts the inexact differential $\,\dbar Q$ into an exact differential. 
So,
\begin{equation}
\frac{\dbar Q}{T} \equiv dS.
\end{equation}
It will be interesting to find out what physical quantities correspond to 
 the functions $T$ and $S$. 

Suppose that the system is thermally insulated, so that $Q=0$. In this case, the
first law of thermodynamics implies that
\begin{equation}
W_{if} = - {\mit\Delta} \bar{E}.
\end{equation}
Thus, in this special case, the work done depends only on the energy 
difference between the initial and final states, and is independent of the process.
In fact, when Clausius first formulated the first law in 1850
this is how he expressed it:
\begin{quote}
{\sf If a thermally isolated system is brought from some initial to some final state
then the work done by the system is independent of the process used.}
\end{quote}

If the external parameters of the system are kept fixed, so that no work is done,
then $\,\dbar W=0$,  Eq.~(\ref{e4.11}) reduces
to
\begin{equation}
\dbar Q = d \bar{E},
\end{equation}
and $\,\dbar Q$ becomes an exact differential. The amount of heat $Q$ absorbed
in going from one macrostate to another 
depends only on the mean energy difference between them, and
 is independent of the process used to effect the change.  In this situation,
heat is a conserved quantity, and acts very much like the invisible indestructible
fluid of Lavoisier's calorific theory.

