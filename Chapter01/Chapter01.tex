\chapter{Introduction}
\section{Intended Audience}
These lecture notes outline a single semester course intended for upper
division undergraduates.

\section{Major Sources}
The textbooks which I have consulted most frequently whilst developing
course material are:
\begin{description}
\item [{\sf Fundamentals of statistical and thermal physics:}] F.~Reif
(McGraw-Hill, New York NY, 1965).
\item [{\sf Introduction to quantum theory:}] D.~Park, 3rd Edition (McGraw-Hill, New York NY, 1992).
\end{description}

\section{Why Study Thermodynamics?}
 Thermodynamics is 
essentially the study  of the internal motions of many body systems. 
Virtually all substances which we encounter in everyday life
 are many body systems of some sort or other ({\em e.g.}, solids, liquids, gases, and light).
Not surprisingly, therefore,
thermodynamics is a discipline with an exceptionally wide range
of applicability. 
Thermodynamics
is certainly the most ubiquitous subfield of physics outside
physics departments. 
Engineers, chemists, and material scientists do not study relativity
or particle physics,
but thermodynamics is an integral, and very important, part of their degree courses. 

Many people are drawn to physics because they want to understand why the
world around us is like it is. For instance, why the sky is blue, why raindrops
are spherical, why we do not fall through the floor, {\em etc}. It turns
out that 
statistical thermodynamics
can explain more things about the world around us than all of the other 
physical theories studied in
the undergraduate 
physics curriculum put together. For instance, in this course we shall explain 
why heat flows from hot to cold bodies, why the air becomes thinner and
colder at higher altitudes,
why the Sun appears yellow 
whereas colder 
stars appear red and hotter stars appear bluish-white, why it is impossible to
measure
a temperature below -273$^\circ$ centigrade, why there is a maximum theoretical
efficiency of a power generation unit which can never be exceeded no matter
what the design, why high mass stars must ultimately collapse
to form black-holes, and much more!

\section{Atomic Theory of Matter}
According to the well-known 
{\em atomic} theory of matter, the familiar objects which 
make up the world around us, such as tables and chairs,
 are themselves made up of a great many 
microscopic particles.

Atomic theory was invented by the ancient Greek philosophers Leucippus
and Democritus, who speculated that the world  essentially consists of
myriads of tiny indivisible particles, which they called {\em atoms}, from the
Greek {\em atomon}, meaning ``uncuttable.''\@ They speculated, further, that
the observable properties of everyday 
materials can be explained either in terms of 
 the
different {\em shapes}\/ of the atoms  which they contain, or the
different {\em motions}\/ of these atoms.
In some respects modern atomic theory differs substantially from the
primitive theory of Leucippus and Democritus, but the central ideas
have remained essentially unchanged. 
In particular, Leucippus and Democritus were right
to suppose that the properties of  materials depend not only
on the nature of the constituent atoms or molecules, 
but also on the relative motions of these particles.

\section{Thermodynamics}
In this  course, we shall focus almost exclusively 
 on those physical properties of everyday materials which are
associated with the {\em motions} of their
constituent atoms or molecules. In particular, we shall be
concerned with the type of motion which we normally
call ``heat.''\@ We shall try to
establish what controls the flow of heat from one body to another
when they are brought into thermal contact. We shall also
attempt to understand the relationship between heat and mechanical work.
For instance, does the heat content of a body  increase when mechanical
work is done on it? More importantly,  can we extract heat from
a body in order to do useful work? This subject area is called
``thermodynamics,'' from the Greek roots {\em thermos}, meaning ``heat,'' and
{\em dynamis}, meaning ``power.''

\section{Need for a Statistical Approach}
It is necessary to emphasize from the 
very outset that this is a {\em difficult}\/
 subject. In fact,
this subject is so difficult that we are forced to adopt a radically different 
approach  to that employed in 
other areas of physics.

 In all of the physics
courses which you have taken up to now, you were eventually
able to formulate some
{\em exact}, or nearly exact, set of equations
 which governed the system under
investigation. For instance, Newton's equations of 
motion, or Maxwell's equations for electromagnetic fields.
You were then  able to analyze
the system by solving 
 these equations, either exactly or approximately. 

In thermodynamics we have no problem formulating the governing
equations. The motions of atoms and molecules are described {\em exactly}
 by the
laws of quantum mechanics. In many cases,  they are also described to a
reasonable  approximation by the much simpler
laws of classical mechanics. We shall not be
dealing with systems sufficiently energetic for  atomic nuclei to be
disrupted, so we can forget about nuclear forces. Also, in general, 
the gravitational forces between  atoms and molecules are completely
negligible. This means that the forces between atoms and molecules are
predominantly electromagnetic in origin,  and
are, therefore, very well understood.  So, in principle, we could write down
the exact laws of motion for a thermodynamical system, including all of
the inter-atomic
forces. The problem is the sheer complexity of this type of
 system. In one mole of
a substance ({\em e.g.},  in twelve grams of carbon, or eighteen grams
of water) there are 
Avagadro's number of atoms or molecules. That is, about
$$
N_A = 6 \times 10^{23}
$$
particles, which is a {\em gigantic}
 number of particles! To solve the system exactly we would
have to write down about $10^{24}$ coupled
equations of motion, with the same number
of initial conditions, and then try to integrate the system. 
Quite plainly,
 this is impossible. It would also be complete overkill. We are not
at  all 
interested in knowing the position and velocity of {\em every}\/ particle in the
system as a function of time.
 Instead, we want to know things like the volume of the system, the
temperature, the pressure, the heat capacity, the coefficient of expansion,
{\em etc}. We would certainly be hard put to specify more than about fifty,
say, properties of  a thermodynamic system in which we are  really interested. 
So, the number of pieces of information we require  
 is absolutely
minuscule compared to the number of degrees of freedom of the system.
 That is,
the number of pieces of information needed to completely specify the 
internal motion.
Moreover, the quantities which we are interested in do not depend on the
motions of individual particles, or some some small subset of  particles,
 but, instead, depend on the average motions of {\em all}
 the particles in the system.
In other words, these quantities depend on the {\em statistical}
 properties of the atomic or molecular motion. 

The method adopted in this subject area is essentially dictated by the 
enormous complexity of thermodynamic systems. We start with some
statistical information about the motions of the constituent atoms or molecules,
 such as their average
kinetic energy, 
 but we possess virtually
no information about the motions of individual particles. We then try
to deduce some other  properties of the system
from a  statistical treatment of the governing equations. 
If fact, our approach 
 {\em has}\/ to be statistical in nature, because we lack most of the
information required to specify the internal state of the system. The best we
can do is to provide a few overall constraints, such as the average volume and
 the
average energy. 

Thermodynamic systems are ideally suited to a statistical approach because of
the enormous numbers of particles they contain. As  you probably know
already, 
statistical arguments actually 
get {\em more exact}\/ as the numbers involved get larger.
For instance, whenever I see an opinion poll published
in a newspaper, I immediately
look at the small print  at the bottom where it says how many people
were interviewed. I know that even if the polling was done without bias,
which is extremely unlikely,
the laws of statistics say that there is a intrinsic error of order 
one over the square root of the number of people questioned.
 It follows that
 if a
 thousand people were interviewed, which is a typical number,
then  the error is at least three percent. Hence, if the headline
 says that so and so
is ahead by one percentage point, and only a thousand people were
polled, then I know the result is statistically meaningless.
We  can easily 
appreciate that if we do statistics on a thermodynamic system containing
$10^{24}$ particles then we are going to  obtain results which are valid to
incredible accuracy. In fact, in most situations we can forget that the
results are statistical at all, and treat them as exact laws of physics.
	For instance, the familiar equation of state of an ideal gas,
$$
P\,V =\nu\, R\,T,
$$
is actually a statistical result. In other words, it relates the {\em
average}\/ pressure and the average volume to the average temperature.
However, for one mole of gas the statistical deviations from average
values are only about $10^{-12}$, according to the $1/\sqrt{N}$ law.
Actually, it is virtually
impossible to  measure  the pressure, volume, or temperature of a gas
to such accuracy, so most people just forget about the fact that the above
expression
is a statistical result, and treat it as a law of physics interrelating
the actual pressure, volume, and temperature of an ideal gas.

\section{Microscopic and Macroscopic Systems}
It is useful, at this stage,
 to make a distinction between the different sizes of the systems
that we are going to  examine. We shall call a system {\em microscopic}\/ if it is
roughly of atomic dimensions, or smaller. On the other hand, we shall call
a system {\em macroscopic}\/ when it is large enough to be visible in the
ordinary sense. This is a rather inexact definition. The exact definition
depends on the number of particles in the system, which we shall call $N$.
A system is macroscopic if
$$
\frac{1}{\sqrt{N}} \ll 1,
$$
which means that statistical arguments can be applied to reasonable accuracy.
For instance, if we wish to keep the statistical error below 
one percent then a macroscopic system would have to contain more than about
ten thousand particles. Any system containing less than this number of
particles would be regarded as essentially microscopic, and, hence,
statistical arguments could not be applied to such a system without
unacceptable error. 

\section{Thermodynamics and Statistical Thermodynamics}
In this course, we are going to develop some machinery for interrelating
the statistical properties of a system containing a very large number of
particles, via a
 statistical treatment of the laws  of atomic or molecular motion.
It turns out that once we have developed this machinery, we can obtain some very
general results which do not depend on the exact details of the statistical
treatment. 
These results can be described without
reference to the underlying statistical nature of the system, but their
validity  depends ultimately on statistical arguments. They
take the form of general statements regarding heat and work, and are 
usually referred to as  {\em classical thermodynamics}, or just 
{\em thermodynamics},
for short. Historically, classical thermodynamics was the first sort of 
thermodynamics to be discovered. In fact, for many years  
the laws of classical
thermodynamics seemed rather mysterious, because their statistical
justification had yet to be discovered.
 The strength of classical
thermodynamics is its great {\em generality}, 
which comes about because it does not
depend on any detailed assumptions about the statistical properties
of the system under investigation. This   generality 
is also the principle weakness of classical thermodynamics. Only a relatively
few statements can be made on such general grounds, so many interesting
properties of the system remain outside the scope of this theory.


If we go beyond classical thermodynamics, and start to investigate the
statistical machinery which underpins it, then we get all of the
results of classical
thermodynamics, plus a large number of other results which enable the 
macroscopic parameters of the system to be calculated from a knowledge
of its microscopic constituents. This approach is known as
{\em statistical thermodynamics}, and is extremely powerful.
 The only drawback 
is that the further we delve inside the statistical machinery
of thermodynamics, the harder it becomes to perform the necessary
calculations.

Note
 that both classical and statistical thermodynamics are
only valid for systems in {\em equilibrium}. 
If the system is not in equilibrium
then the problem becomes considerably more difficult. 
In fact, the thermodynamics of 
non-equilibrium systems, which
 is generally called {\em irreversible thermodynamics}, 
is a graduate level subject.

\section{Classical and Quantum Approaches}
We mentioned earlier that the 
motions (by which we really meant the translational motions) 
of atoms and molecules are described exactly
by quantum mechanics, and only  approximately by classical mechanics.
It turns out that the non-translational motions 
of molecules, such as their rotation
and vibration, are  very poorly described by classical mechanics.
So, why bother using classical mechanics at all? Unfortunately, quantum
mechanics deals with the translational motions of atoms and
molecules (via wave mechanics) in a rather awkward manner. The
classical approach is far more straightforward, and, under most
circumstances, yields the same statistical results. Hence, in the
bulk of this course,
we shall use classical mechanics, as much
as possible, to describe translational motions, and reserve quantum
mechanics for dealing with  non-translational motions. However, towards
the end of this course, we shall switch to a purely quantum mechanical
approach.


