\chapter{Quantum Statistics}

\section{Introduction}
Previously, we investigated the statistical thermodynamics  of ideal gases
using a rather {\em ad hoc}\/ combination of classical and
quantum mechanics (see Sects.~\ref{s7.8} and
\ref{s7.9}). In fact, we employed classical mechanics to deal with
the translational degrees of freedom of the constituent particles,  and quantum
mechanics to deal with the non-translational degrees of freedom. Let us
now discuss ideal gases from a purely quantum mechanical standpoint. It turns
out that this approach is necessary to deal with either low temperature or high
density gases. Furthermore, it also allows us to investigate 
completely non-classical
``gases,'' such as photons or the conduction electrons in a metal.

\section{Symmetry Requirements in Quantum Mechanics}
Consider a gas consisting of $N$ {\em identical}, non-interacting, structureless
particles enclosed within
a container of volume $V$. Let $Q_i$ denote collectively
all the coordinates of the $i$th particle: {\em i.e.}, the three
Cartesian coordinates which determine its
spatial position, as well as the  spin coordinate which determines its
internal state. Let $s_i$ be an index labeling the possible quantum states
of the $i$th particle: {\em i.e.}, each possible value of $s_i$ corresponds
to a specification of the three momentum components of the particle, as well
as the direction of its spin orientation. According to quantum mechanics,
the overall 
state of the system when the $i$th particle is in state $s_i$, {\em etc}.,
is {\em completely determined}\/ by the complex {\em wave-function}\/  
\begin{equation}
{\mit\Psi}_{s_1,\cdots,s_N} (Q_1, Q_2, \cdots, Q_N).
\end{equation}
In particular, the probability of an observation of the system
finding the $i$th particle with coordinates in the range $Q_i$ to $Q_i+dQ_i$,
{\em etc}.,
is simply
\begin{equation}
|{\mit\Psi}_{s_1,\cdots,s_N} (Q_1, Q_2, \cdots, Q_N)|^2\,dQ_1\,dQ_2\,\cdots\,dQ_N.
\end{equation}

One of the fundamental postulates of quantum mechanics is the essential
{\em indistinguishability}\/ of particles of the same species. What this
means, in practice, is that we cannot {\em label}\/ particles of the same
species:
 {\em i.e.}, a proton is just a proton---we cannot meaningfully  talk of proton number 1 and proton number 2, {\em etc}.
Note that no such constraint arises in classical mechanics. Thus, in
classical mechanics particles of the
same species are regarded as being {\em distinguishable}, 
and can, therefore, be labelled. Of course, the quantum mechanical approach is the correct one.

Suppose that we {\em interchange}\/ the $i$th and $j$th particles:
{\em i.e.},
\begin{eqnarray}
Q_i &\leftrightarrow & Q_j,\\[0.5ex]
s_i & \leftrightarrow & s_j.
\end{eqnarray}
 If the particles
are truly indistinguishable then nothing has changed: 
{\em i.e.}, we have a particle
in quantum state $s_i$ and a particle in quantum state $s_j$ both before
and after the particles are swapped. Thus, the probability of observing the
system in a given state also cannot have changed: {\em i.e.},
\begin{equation}
|{\mit\Psi}(\cdots Q_i\cdots Q_j\cdots)|^2=
|{\mit\Psi}(\cdots Q_j\cdots Q_i\cdots)|^2.\label{e8.5}
\end{equation}
Here, we have omitted the subscripts $s_1,\cdots,s_N$ for the sake of
clarity.
Note that we cannot conclude that the wave-function ${\mit \Psi}$ is unaffected
when the particles are swapped, because ${\mit\Psi}$ cannot be observed
experimentally. Only the {\em probability density} $|{\mit\Psi}|^2$ is observable.
Equation~(\ref{e8.5}) implies that
\begin{equation}\label{e8.6}
{\mit\Psi}(\cdots Q_i\cdots Q_j\cdots)=A\,{\mit\Psi}(\cdots Q_j\cdots Q_i\cdots),
\end{equation}
where $A$ is a complex constant of modulus unity: {\em i.e.}, $|A|^2=1$. 

Suppose that we interchange  the $i$th and $j$th particles a second time. 
Swapping the $i$th and $j$th particles twice leaves the system completely
unchanged: {\em i.e.}, it is equivalent to doing nothing to the system.
Thus, the wave-functions before and after this process must be identical.
It follows from Eq.~(\ref{e8.6}) that
\begin{equation}
A^2 = 1.
\end{equation}
Of course, the only solutions to the above equation are $A=\pm 1$. 

We conclude, from the above discussion, that the wave-function ${\mit\Psi}$ is
either completely {\em symmetric}\/ under the interchange of particles, or
it is completely {\em anti-symmetric}. In other words,
either
\begin{equation}
{\mit\Psi}(\cdots Q_i\cdots Q_j\cdots)=
+{\mit\Psi}(\cdots Q_j\cdots Q_i\cdots),\label{e8.8}
\end{equation}
or 
\begin{equation}
{\mit\Psi}(\cdots Q_i\cdots Q_j\cdots)=
-{\mit\Psi}(\cdots Q_j\cdots Q_i\cdots).\label{e8.9}
\end{equation}

In 1940 the Nobel prize winning physicist Wolfgang Pauli demonstrated,
via arguments involving  relativistic invariance, that
the wave-function associated with  a collection of 
identical integer-spin ({\em i.e.}, spin $0, 1, 2$, {\em etc}.)
particles satisfies Eq.~(\ref{e8.8}), whereas the wave-function
associated with a collection of identical half-integer-spin
({\em i.e.}, spin $1/2, 3/2, 5/2$, {\em etc}.)
particles satisfies Eq.~(\ref{e8.9}). The former type of particles
are known as {\em bosons}\/ [after the Indian physicist S.N.~Bose, who
first put forward Eq.~(\ref{e8.8}) on empirical grounds].
The latter type of particles are called  {\em fermions}\/ (after the Italian
physicist  Enrico Fermi, who first studied the properties of
 fermion gases). Common examples of bosons are photons and ${\rm He}^4$ atoms.
Common examples of fermions are protons, neutrons, and electrons. 

Consider a gas made up of
identical bosons. Equation~(\ref{e8.8}) implies that the interchange of any  
two particles
does not lead to a new state of the system. Bosons must, therefore, be
considered as genuinely indistinguishable when enumerating the different possible
states of the gas. Note that Eq.~(\ref{e8.8}) imposes no restriction on how many
particles can occupy a given single-particle quantum state $s$. 

Consider a gas made up of identical fermions. Equation~(\ref{e8.9}) 
implies that the interchange of any two particles does not
lead to a new physical state of the system (since $|{\mit\Psi}|^2$
is invariant). Hence, fermions must also be
considered  genuinely indistinguishable when enumerating the different possible
states of the gas. Consider the special case where particles $i$ and $j$ lie in
the same quantum state. In this case, the act of swapping the two
particles is equivalent to leaving the system unchanged, so
\begin{equation}
{\mit\Psi}(\cdots Q_i\cdots Q_j\cdots)=
{\mit\Psi}(\cdots Q_j\cdots Q_i\cdots). \label{e8.10}
\end{equation}
However, Eq.~(\ref{e8.9}) is also applicable, since the two particles
are fermions. The only way in which Eqs.~(\ref{e8.9}) and (\ref{e8.10}) can
be reconciled is if
\begin{equation}
{\mit\Psi} = 0
\end{equation}
wherever particles $i$ and $j$ lie in the same quantum state. This
is another way of saying that it is  {\em impossible}\/ for any two
particles  in a gas of  fermions to lie
 in the same single-particle quantum state. This proposition is known
as the {\em Pauli exclusion principle}, since it was first proposed
by W.~Pauli in 1924 on empirical grounds. 

Consider, for the sake of comparison, a gas made up of identical
classical particles. In this case, the particles must be considered
distinguishable when enumerating the different possible
states of the gas. Furthermore, there are no constraints on 
how many particles can occupy a
 given quantum state.

According to the above discussion, there are {\em three}\/  different sets of rules
which can be  used to enumerate the states of a gas made up of identical
particles. For a boson gas, the particles must be treated as
being indistinguishable, and there is no limit to how many particles
can occupy a given quantum state. This set of rules is called {\em Bose-Einstein
statistics}, after S.N.~Bose and A.~Einstein, who first developed them.
For a fermion gas, the particles must be treated as
being indistinguishable, and there can never be more than one particle in
any given quantum state. This set of rules is called {\em Fermi-Dirac
statistics}, after E.~Fermi and P.A.M.~Dirac, who first developed them.
Finally, for a classical gas, the particles 
 must be treated as
being distinguishable, and there is no limit to how many particles
can occupy a given quantum state. This set of rules is called
{\em Maxwell-Boltzmann statistics}, after J.C.~Maxwell and L.~Boltzmann,
who first developed them.

\section{Illustrative Example}
Consider a very simple gas made up of two identical particles. Suppose
that each particle can be in one of three possible quantum states, $s=1,2,3$. 
Let us enumerate the possible states of the whole gas according to
Maxwell-Boltzmann, Bose-Einstein, and Fermi-Dirac statistics, respectively.

For the case of Maxwell-Boltzmann (MB) statistics, the two particles are
considered to be distinguishable. Let us denote them $A$ and $B$.
Furthermore, any number of particles can occupy the same quantum state. 
The possible different states of the gas are shown in Tab.~\ref{tmb}.
\begin{table}[h]\centering
\begin{tabular}{ccc}\hline 1&2&3 \\\hline
$AB$     & $\cdots$ & $\cdots$ \\
$\cdots$ & $AB$     & $\cdots$ \\
$\cdots$ & $\cdots$ & $AB$     \\
$A$      & $B$      & $\cdots$ \\
$B$      & $A$      & $\cdots$ \\
$A$      & $\cdots$ & $B$      \\
$B$      & $\cdots$ & $A$      \\
$\cdots$ & $A$      & $B$      \\
$\cdots$ & $B$      & $A$      \\\hline
\end{tabular}
\caption{\em Two particles distributed amongst three states according to
Maxwell-Boltzmann statistics.}\label{tmb}
\end{table}
There are clearly 9 distinct states.

For the case of Bose-Einstein (BE) statistics, the two particles are
considered to be indistinguishable. Let us denote them both as $A$.
Furthermore, any number of particles can occupy the same quantum state. 
The possible different states of the gas are shown in Tab.~\ref{tbe}.
\begin{table}[h]\centering
\begin{tabular}{ccc}\hline 1&2&3 \\\hline
$AA$     & $\cdots$ & $\cdots$ \\
$\cdots$ & $AA$     & $\cdots$ \\
$\cdots$ & $\cdots$ & $AA$     \\
$A$      & $A$      & $\cdots$ \\
$A$      & $\cdots$ & $A$      \\
$\cdots$ & $A$      & $A$      \\\hline
\end{tabular}
\caption{\em Two particles distributed amongst three states according to
Bose-Einstein statistics.}\label{tbe}
\end{table}
There are clearly 6 distinct states.

Finally, for the case of Fermi-Dirac (FD) statistics, the two particles are
considered to be indistinguishable. Let us again denote them both as $A$.
Furthermore, no more than one particle  can occupy a given quantum state. The possible different states of the gas are shown in Tab.~\ref{tfd}.
\begin{table}[h]\centering
\begin{tabular}{ccc}\hline 1&2&3 \\\hline
$A$      & $A$      & $\cdots$ \\
$A$      & $\cdots$ & $A$      \\
$\cdots$ & $A$      & $A$      \\\hline
\end{tabular}
\caption{\em Two particles distributed amongst three states according to
Fermi-Dirac statistics.}\label{tfd}
\end{table}
There are clearly only 3  distinct states.

It follows, from the above example, that Fermi-Dirac statistics are
more restrictive ({\em i.e.}, there are less possible states of the
system) than Bose-Einstein statistics, which are, in turn, more restrictive
than Maxwell-Boltzmann statistics. Let
\begin{equation}
\xi \equiv \frac{\mbox{ probability that the two particles are found in the
same state}}{\mbox{ probability that the two particles are found in different
states}}.
\end{equation}
For the case under investigation,
\begin{eqnarray}
\xi_{\rm MB} &=& 1/2,\\[0.5ex]
\xi_{\rm BE} &=& 1,\\[0.5ex]
\xi_{\rm FD} &=& 0.
\end{eqnarray}
We conclude that in Bose-Einstein statistics there is a greater relative
tendency for particles to cluster in the same state than in classical
statistics. On the other hand, in Fermi-Dirac statistics there is
less tendency for particles to cluster in the same state than
in classical statistics.

\section{Formulation of the statistical problem}
Consider a gas consisting of $N$ identical non-interacting particles
occupying volume $V$ and in thermal equilibrium at temperature $T$. 
Let us label the possible quantum states of a single particle by $r$ (or $s$). 
Let the energy of a particle in state $r$ be denoted $\epsilon_r$. 
Let the number of particles in state $r$ be written $n_r$. Finally,
let us label the possible quantum states of the whole gas by $R$.

The particles are assumed to be non-interacting, so  the
total energy of the gas in state $R$, where there are $n_r$ particles
in quantum state $r$, {\em etc.}, is simply
\begin{equation}
E_R = \sum_r\,n_r\,\epsilon_r,
\end{equation}
where the sum extends over all possible quantum states $r$. Furthermore,
since the total number of particles in the gas is known to be $N$, we must
have
\begin{equation}
N = \sum_r n_r.
\end{equation}

In order to calculate the thermodynamic properties of the gas ({\em i.e.}, 
its internal energy  or its entropy), it is necessary to
calculate its {\em partition function},
\begin{equation}
Z =\sum_R {\rm e}^{-\beta\,E_R} = \sum_R {\rm e}^{-\beta\,(n_1\,\epsilon_1+n_2\,
\epsilon_2+\cdots)}.\label{e8z}
\end{equation}
Here, the sum is over all possible states $R$ of the whole gas:
{\em i.e.}, over all the various possible values of the
numbers $n_1, n_2,\cdots$. 

Now, $\exp[-\beta\,(n_1\,\epsilon_1+n_2\,\epsilon_2+\cdots)]$ is
the relative probability of finding the gas in a particular state in which
there are $n_1$ particles in state 1, $n_2$ particles in state 2, {\em etc}. 
Thus, the mean number of particles in quantum state $s$ can be written
\begin{equation}\label{e8z1}
\bar{n}_s = \frac{\sum_R n_s\,\exp[-\beta\,(n_1\,\epsilon_1+n_2\,\epsilon_2+\cdots)]}
{\sum_R \exp[-\beta\,(n_1\,\epsilon_1+n_2\,\epsilon_2+\cdots)]}.
\end{equation}
A comparison of Eqs.~(\ref{e8z}) and (\ref{e8z1})  yields the result
\begin{equation}\label{e820}
\bar{n}_s = -\frac{1}{\beta}\frac{\partial \ln Z}{\partial\epsilon_s}.
\end{equation}
Here, $\beta\equiv 1/k\,T$.

\section{Fermi-Dirac Statistics}\label{s8fd}
Let us, first of all, consider Fermi-Dirac statistics. According to Eq.~(\ref{e8z1}),
the average number of particles in quantum state $s$ can be written
\begin{equation}
\bar{n}_s = \frac{\sum_{n_s} n_s\,{\rm e}^{-\beta\,n_s\,\epsilon_s}\,
\sum_{n_1,n_2,\cdots}^{(s)} {\rm e}^{-\beta\,(n_1\,\epsilon_1+n_2\,\epsilon_2+
\cdots)}}{\sum_{n_s} {\rm e}^{-\beta\,n_s\,\epsilon_s}\,
\sum_{n_1,n_2,\cdots}^{(s)} {\rm e}^{-\beta\,(n_1\,\epsilon_1+n_2\,\epsilon_2+
\cdots)}}.\label{ens}
\end{equation}
Here, we have rearranged the order of summation, using the multiplicative
properties of the exponential function. Note that the first sums in the
numerator and denominator
only involve $n_s$, whereas the last sums
 omit the particular state $s$ from consideration
(this is indicated by the superscript $s$ on the summation symbol). Of course,
the sums in the above expression range over all values of the
numbers $n_1, n_2, \cdots$ such that $n_r=0$ and 1 for
each $r$, subject to the overall constraint that
\begin{equation}
\sum_r n_r = N.
\end{equation}

Let us introduce the function
\begin{equation}\label{e8zs}
Z_s(N) = \sum_{n_1,n_2,\cdots}^{(s)}
{\rm e}^{-\beta\,(n_1\,\epsilon_1+n_2\,\epsilon_2+
\cdots)},
\end{equation}
which is defined as the partition function for $N$ particles distributed over all
quantum states, {\em excluding}\/ state $s$, according to Fermi-Dirac statistics.
By explicitly performing the sum over $n_s=0$ and 1, the
expression (\ref{ens}) reduces to
\begin{equation}\label{e8da}
\bar{n}_s = \frac{0 + {\rm e}^{-\beta\,\epsilon_s}\,Z_s(N-1)}
{Z_s(N) + {\rm e}^{-\beta\,\epsilon_s}\,Z_s(N-1)},
\end{equation}
which yields
\begin{equation}\label{e8dz}
\bar{n}_s =\frac{1}{[Z_s(N)/Z_s(N-1)]\,{\rm e}^{\,\beta\,\epsilon_s} + 1}.
\end{equation}


In order to make further progress, we must somehow relate $Z_s(N-1)$ to
$Z_s(N)$. Suppose that ${\mit\Delta}N\ll N$. It follows that 
$\ln Z_s(N-{\mit\Delta}N)$ can be Taylor expanded  to give
\begin{equation}\label{e8dc}
\ln Z_s(N-{\mit\Delta}N) \simeq \ln Z_s(N) - \frac{\partial \ln Z_s}{\partial N}\,
{\mit\Delta} N = \ln Z_s(N) - \alpha_s\,{\mit\Delta}N,
\end{equation}
where 
\begin{equation}
\alpha_s\equiv \frac{\partial \ln Z_s}{\partial N}.\label{e8de}
\end{equation}
As always, we Taylor expand the slowly varying function $\ln Z_s(N)$, rather
than the rapidly varying function $Z_s(N)$, because the radius of
convergence of the latter Taylor series is too small for the series to
be of any practical use. Equation~(\ref{e8dc}) can be rearranged to give
\begin{equation}\label{e8db}
Z_s(N-{\mit\Delta}N) = Z_s(N)\,{\rm e}^{-\alpha_s\,{\mit\Delta}N}.
\end{equation}

Now, since $Z_s(N)$ is a sum over very many
different quantum states, we would not expect the logarithm of this
function to be  sensitive to which particular state $s$
is  excluded from consideration. Let us, therefore, introduce the
approximation that $\alpha_s$ is independent of $s$, so that we can write
\begin{equation}\label{e8df}
\alpha_s \simeq \alpha
\end{equation}
for all $s$. It follows that the derivative (\ref{e8de}) can be expressed
approximately in terms of the derivative of the full partition function
$Z(N)$ (in which the $N$ particles are distributed over
{\em all}\/ quantum states). In fact,
\begin{equation}\label{e830}
\alpha\simeq \frac{\partial \ln Z}{\partial N}.
\end{equation}

Making use of Eq.~(\ref{e8db}), with ${\mit\Delta} N =1$, plus the
approximation (\ref{e8df}), the expression (\ref{e8dz}) reduces to
\begin{equation}
\bar{n}_s = \frac{1}{{\rm e}^{\,\alpha+\beta\,\epsilon_s}+ 1}.\label{efd}
\end{equation}
This is called the {\em Fermi-Dirac distribution}. The parameter $\alpha$
is determined by the constraint that $\sum_r\bar{n}_r = N$: {\em i.e.}, 
\begin{equation}
\sum_r \frac{1}{{\rm e}^{\,\alpha+\beta\,\epsilon_r}+ 1} = N.
\end{equation}


Note that $\bar{n}_s\rightarrow 0$ if $\epsilon_s$ becomes sufficiently 
large.
On the other hand, since the denominator in Eq.~(\ref{efd}) can never become
less than unity, no matter how small $\epsilon_s$ becomes, it follows
that $\bar{n}_s\leq 1$. Thus,
\begin{equation} 
0 \leq \bar{n}_s \leq 1,
\end{equation}
in accordance with the Pauli exclusion principle.

Equations (\ref{e820}) and (\ref{e830}) can be integrated to
give
\begin{equation}
\ln Z = \alpha\,N + \sum_r\ln \,(1+{\rm e}^{-\alpha-\beta\,\epsilon_r}),
\end{equation}
where use has been made of Eq.~(\ref{efd}).

\section{Photon Statistics}
Up to now, we have assumed that the number of particles $N$ contained
in a given system is a fixed
number. This is a reasonable assumption if the particles possess non-zero
mass, since we are not generally considering 
relativistic systems in this  course.
However, this assumption breaks down for the case of photons, which are
{\em zero-mass}\/ bosons. In fact,   photons enclosed in a container
of volume $V$, maintained at temperature $T$, can readily be absorbed
or emitted by the walls. Thus, for the special case of a gas of
photons there is no requirement which limits the total number of particles.

It follows, from the above discussion, that photons obey a simplified
form of Bose-Einstein statistics in which there is an unspecified
total number of particles. This type of statistics is called {\em photon
statistics}.

Consider the expression (\ref{ens}). For the case of photons, the
numbers $n_1, n_2,\cdots$ assume all values $n_r=0,1,2,\cdots$ for each $r$,
without any further restriction. It follows that the sums $\sum^{(s)}$ in the
numerator and denominator are identical and, therefore, cancel. Hence,
Eq.~(\ref{ens}) reduces to
\begin{equation}
\bar{n}_s = \frac{\sum_{n_s} n_s\,{\rm e}^{-\beta\,n_s\,\epsilon_s}}
{\sum_{n_s} {\rm e}^{-\beta\,n_s\,\epsilon_s}}.\label{e8sum}
\end{equation}
However, the above expression can be rewritten
\begin{equation}
\bar{n}_s =-\frac{1}{\beta}\frac{\partial}{\partial \epsilon_s}\left(\ln\sum_{n_s}
{\rm e}^{-\beta\,n_s\,\epsilon_s}\right).\label{e8zz}
\end{equation}
Now, the sum on the right-hand side of the above equation is an infinite
geometric series, which can easily be evaluated. In fact,
\begin{equation}
\sum_{n_s=0}^\infty {\rm e}^{-\beta\,n_s\,\epsilon_s} 
= 1 + {\rm e}^{-\beta\,\epsilon_s} + {\rm e}^{-2\,\beta\,\epsilon_s} +
\cdots = \frac{1}{1-{\rm e}^{-\beta\,\epsilon_s}}.
\end{equation}
Thus, Eq.~(\ref{e8zz}) gives
\begin{equation}
\bar{n}_s =\frac{1}{\beta} \frac{\partial}{\partial\epsilon_s}
\ln\,(1-{\rm e}^{-\beta\,\epsilon_s}) =\frac{{\rm e}^{-\beta\,\epsilon_s}}
{1-{\rm e}^{-\beta\,\epsilon_s}},
\end{equation}
or
\begin{equation}\label{e8zzz}
\bar{n}_s = \frac{1}{{\rm e}^{\,\beta\,\epsilon_s}-1}.
\end{equation}
This is known as the {\em Planck distribution}, after the German physicist
Max Planck who first proposed it in 1900 on purely empirical grounds.

Equation (\ref{e820}) can be integrated to
give
\begin{equation}
\ln Z = -\sum_r \ln\,(1-{\rm e}^{-\beta\,\epsilon_r}),
\end{equation}
where use has been made of Eq.~(\ref{e8zzz}).

\section{Bose-Einstein Statistics}
Let us now consider Bose-Einstein statistics. The particles in the
system are assumed to be massive, so  the total number of
particles $N$ is a fixed number. 

Consider the expression (\ref{ens}). For the case of massive bosons, the numbers
$n_1,n_2, \cdots$ assume all values $n_r=0,1,2,\cdots$ for each $r$, subject to
the constraint that $\sum_r n_r=N$. 
Performing explicitly the sum over $n_s$, this expression reduces to
\begin{equation}
\bar{n}_s=\frac{0 +{\rm e}^{-\beta\,\epsilon_s}\,Z_s(N-1)
+ 2\,{\rm e}^{-2\,\beta\,\epsilon_s}\,Z_s(N-2)+\cdots}
{Z_s(N) +{\rm e}^{-\beta\,\epsilon_s}\,Z_s(N-1)
+ {\rm e}^{-2\,\beta\,\epsilon_s}\,Z_s(N-2)+\cdots},
\end{equation}
where $Z_s(N)$ is the partition function for $N$ particles
distributed over all quantum states, {\em excluding}\/ state $s$, according
to Bose-Einstein statistics [{\em cf.}, Eq.~(\ref{e8zs})]. Using Eq.~(\ref{e8db}),
and the approximation (\ref{e8df}), the above equation reduces to
\begin{equation}
\bar{n}_s = \frac{\sum_s n_s\,{\rm e}^{-n_s\,(\alpha+\beta\,\epsilon_s)}}
{\sum_s {\rm e}^{-n_s\,(\alpha+\beta\,\epsilon_s)}}.
\end{equation}
Note that this expression is identical to (\ref{e8sum}), except that
$\beta\,\epsilon_s$ is replaced by $\alpha+\beta\,\epsilon_s$. Hence,
an analogous calculation to that outlined in the previous subsection yields
\begin{equation}
\bar{n}_s = \frac{1}{{\rm e}^{\,\alpha+\beta\,\epsilon_s}-1}.\label{e8ddd}
\end{equation}
This is called the {\em Bose-Einstein distribution}. Note that $\bar{n}_s$
can become very large in this distribution. The parameter $\alpha$ is again determined
by the constraint on the total number of particles: {\em i.e.},
\begin{equation}
\sum_r
\frac{1}{{\rm e}^{\,\alpha+\beta\,\epsilon_r}-1} =N.\label{e8eee}
\end{equation}


Equations (\ref{e820}) and (\ref{e830}) can be integrated to
give
\begin{equation}
\ln Z = \alpha\,N - \sum_r\ln \,(1-{\rm e}^{-\alpha-\beta\,\epsilon_r}),
\end{equation}
where use has been made of Eq.~(\ref{e8ddd}).

Note that photon statistics correspond to the special case of Bose-Einstein
statistics in which the parameter $\alpha$ takes the value zero,  and the constraint
(\ref{e8eee}) does not apply.

\section{Maxwell-Boltzmann Statistics}
For the purpose of comparison, it is instructive to consider the purely
classical case of Maxwell-Boltzmann statistics. The partition
function is written
\begin{equation}\label{e846}
Z=\sum_{R} {\rm e}^{-\beta\,(n_1\,\epsilon_1+n_2\,\epsilon_2+\cdots)},
\end{equation}
where the sum is over all distinct states $R$ of the gas, and the particles
are treated as distinguishable. For given values of $n_1, n_2,\cdots$ there
are
\begin{equation}
\frac{N!}{n_1 !\,n_2!\,\cdots}
\end{equation}
possible ways in which $N$ distinguishable 
particles can be put into individual quantum states such that there are $n_1$
particles in state 1, $n_2$ particles in state 2, {\em etc}. Each of these possible
arrangements corresponds to a distinct state for the whole gas.
Hence, Eq.~(\ref{e846}) can be written 
\begin{equation}\label{e847}
Z = \sum_{n_1,n_2,\cdots} \frac{N!}{n_1 !\,n_2!\,\cdots} {\rm e}^{-\beta\,(n_1\,\epsilon_1+n_2\,\epsilon_2+\cdots)},
\end{equation}
where the sum is over all values of $n_r=0,1,2,\cdots$ for each $r$, subject to
the constraint that
\begin{equation}\label{e848}
\sum_r n_r = N.
\end{equation}
Now, Eq.~(\ref{e847}) can be written
\begin{equation}
Z = \sum_{n_1,n_2,\cdots} \frac{N!}{n_1 !\,n_2!\,\cdots} 
({\rm e}^{-\beta\,\epsilon_1})^{n_1}\,({\rm e}^{-\beta\,\epsilon_2})^{n_2}
\cdots,
\end{equation}
which, by virtue of Eq.~(\ref{e848}), is just the result of
expanding a polynomial. In fact,
\begin{equation}
Z = ({\rm e}^{-\beta\,\epsilon_1}+{\rm e}^{-\beta\,\epsilon_2}+
\cdots)^N,
\end{equation}
or
\begin{equation}\label{e849}
\ln Z = N\,\ln\!\left(\sum_r {\rm e}^{-\beta\,\epsilon_r}\right).
\end{equation}
Note that the argument of the logarithm is simply the partition function
for a single particle. 

Equations~(\ref{e820}) and (\ref{e849}) can be combined to give
\begin{equation}
\bar{n}_s = N\,\frac{{\rm e}^{-\beta\,\epsilon_s}}
{\sum_r {\rm e}^{-\beta\,\epsilon_r}}.
\end{equation}
This is known as the {\em Maxwell-Boltzmann distribution}. It is, of course,
just the result obtained by applying the Boltzmann distribution to a single
particle (see Sect.~\ref{s7}).

\section{Quantum Statistics in Classical Limit}
The preceding analysis regarding the quantum statistics of ideal
gases is summarized in the following statements. The mean number of
particles occupying quantum state $s$ is given by
\begin{equation}\label{e8.54}
\bar{n}_s = \frac{1}{{\rm e}^{\,\alpha+\beta\,\epsilon_s} \pm 1},
\end{equation}
where the upper sign corresponds to Fermi-Dirac statistics and the
lower sign corresponds to Bose-Einstein statistics. The parameter
$\alpha$ is determined via
\begin{equation}\label{e8.55}
\sum_r \bar{n}_r = \sum_r \frac{1}{{\rm e}^{\,\alpha+\beta\,\epsilon_r} \pm 1}
=N.
\end{equation}
Finally, the partition function of the gas is given by
\begin{equation}\label{e8.56}
\ln Z = \alpha\,N \pm \sum_r \ln\,(1\pm {\rm e}^{-\alpha-\beta\,\epsilon_r}).
\end{equation}

Let us investigate the magnitude of $\alpha$ in some important limiting
cases. Consider, first of all, the case of a gas at a given temperature
when its concentration is made sufficiently low: {\em i.e.}, when
$N$ is made sufficiently small. The relation (\ref{e8.55}) can only
be satisfied if each term in the sum over states is made 
sufficiently small; {\em i.e.}, if $\bar{n}_r\ll 1$ or $\exp\,(\alpha +
\beta\,\epsilon_r)\gg 1$ for all states $r$. 

Consider, next, the case of a gas made up of a fixed number of particles
when its temperature is made sufficiently large: {\em i.e.}, when $\beta$ is
made sufficiently small. In the sum in Eq.~(\ref{e8.55}), the terms of
appreciable magnitude are those for which $\beta\,\epsilon_r\ll \alpha$. 
Thus, it follows that as $\beta\rightarrow 0$ an increasing number of
terms with large values of $\epsilon_r$ contribute substantially to this
sum. In order to prevent the sum from exceeding $N$, the parameter $\alpha$
must become large enough that each term is made sufficiently small: {\em i.e.},
it is again necessary that $\bar{n}_r\ll 1$ or  $\exp\,(\alpha +
\beta\,\epsilon_r)\gg 1$ for all states $r$. 

The above discussion suggests that if the concentration of an ideal
gas is made sufficiently low, or the temperature is made sufficiently high,
then $\alpha$ must become so large that
\begin{equation}
{\rm e}^{\,\alpha+\beta\,\epsilon_r}\gg 1\label{e8c1}
\end{equation}
for all $r$. Equivalently, this means that the  number of particles occupying
each quantum state must
become so small that
\begin{equation}
\bar{n}_r \ll 1\label{e8c2}
\end{equation}
for all $r$. It is conventional to refer to the limit of sufficiently
low concentration, or sufficiently high temperature, in which Eqs.~(\ref{e8c1})
and Eqs.~(\ref{e8c2}) are satisfied, as the {\em classical limit}.

According to Eqs.~(\ref{e8.54}) and (\ref{e8c1}), both
the  Fermi-Dirac and Bose-Einstein
distributions reduce to
\begin{equation}
\bar{n}_s = {\rm e}^{-\alpha-\beta\,\epsilon_s}
\end{equation}
in the classical limit, whereas the constraint (\ref{e8.55}) yields
\begin{equation}\label{e888}
\sum_r {\rm e}^{-\alpha-\beta\,\epsilon_r} = N.
\end{equation}
The above expressions can be combined to give
\begin{equation}
\bar{n}_s = N\,
 \frac{{\rm e}^{-\beta\,\epsilon_s}}{\sum_r {\rm e}^{-\beta\,\epsilon_r}}.
\end{equation}
It follows that in the classical limit of sufficiently low density,
or sufficiently high temperature, the quantum distribution functions,
whether Fermi-Dirac or Bose-Einstein, reduce to the Maxwell-Boltzmann
distribution. It is easily demonstrated that the physical criterion for the
validity of the classical approximation is that the mean separation between
particles should be much greater than their mean de~Broglie wavelengths.

Let us now consider the behaviour of the partition function (\ref{e8.56})
in the classical limit. We can expand the logarithm to
give
\begin{equation}
\ln Z = \alpha\,N \pm \sum_r\left(\pm {\rm e}^{-\alpha-\beta\,\epsilon_r}\right)
=\alpha\,N +N.
\end{equation}
However, according to Eq.~(\ref{e888}),
\begin{equation}
\alpha =-\ln N + \ln \left({\sum_r {\rm e}^{-\beta\,\epsilon_r}}\right).
\end{equation}
It follows that
\begin{equation}
\ln Z = -N\,\ln N + N +N\,\ln\! \left({\sum_r {\rm e}^{-\beta\,\epsilon_r}}\right).
\end{equation}
Note that this {\em does not}\/ equal the partition function $Z_{\rm MB}$
computed in Eq.~(\ref{e849}) from Maxwell-Boltzmann statistics: {\em i.e.}, 
\begin{equation}
\ln Z_{\rm MB} = N\,\ln\!\left(\sum_r {\rm e}^{-\beta\,\epsilon_r}\right).
\end{equation}
In fact,
\begin{equation}
\ln Z = \ln Z_{\rm MB} - \ln N!,
\end{equation}
or
\begin{equation}
Z = \frac{Z_{\rm MB}}{N!},
\end{equation}
where use has been made of Stirling's approximation ($N!\simeq N\,\ln N - N$),
since $N$ is large. Here, the factor $N!$ simply corresponds to the number
of different permutations of the $N$ particles: permutations which are
physically meaningless when the particles are identical. Recall, that
we had to introduce precisely this factor, in an {\em ad hoc}\/ fashion,
in Sect.~\ref{s7.9} in order to avoid the non-physical consequences of
the Gibb's paradox. Clearly, there is no Gibb's paradox when an ideal
gas is treated properly via quantum mechanics. 

In the classical limit,
a full quantum mechanical analysis of an ideal gas reproduces the results
obtained in Sects.~\ref{s7.8} and \ref{s7.9}, except that the
arbitrary parameter $h_0$ is replaced by Planck's constant
$h=6.61\times 10^{-34}\,{\rm J\,s}$.  

A gas in the classical limit, where the typical de~Broglie wavelength of the
constituent particles is  much smaller than the typical inter-particle
spacing, is said to be {\em non-degenerate}. In the opposite limit,
where the concentration and temperature are such that the typical
de~Broglie wavelength 
becomes comparable with the typical inter-particle spacing, and the actual
Fermi-Dirac or Bose-Einstein distributions must be employed, the
gas is said to be {\em degenerate}.

\section{Planck Radiation Law}
Let us now consider the application of statistical thermodynamics to electromagnetic
radiation. According to Maxwell's theory, an electromagnetic wave is a coupled
self-sustaining 
oscillation of electric and magnetic fields which propagates though a vacuum at
the speed of light, $c=3\times 10^8\,{\rm m}\,{\rm s}^{-1}$. The electric
 component of the wave  can be written
\begin{equation}
{\bf E} = {\bf E}_0 \exp[\,{\rm i}\,({\bf k}\!\cdot \!{\bf r} - \omega\, t)],
\end{equation}
where ${\bf E}_0$ is a constant, ${\bf k}$ is the wave-vector which determines
the wavelength and direction of propagation of the wave, and $\omega$ is
the frequency. The dispersion relation
\begin{equation}
\omega = k\, c\label{e8.69}
\end{equation}
ensures that the wave propagates at the speed of light. Note that this dispersion
relation is very similar to that of sound waves in solids [see Eq.~(\ref{e7.146})].
Electromagnetic waves always propagate in the direction perpendicular to the
coupled electric and magnetic fields ({\em i.e.}, 
electromagnetic waves are transverse waves).
This means that ${\bf k}\!\cdot\! {\bf E}_0 = 0$. Thus, once ${\bf k}$ is specified,
there are only {\em two}\/ possible
independent
 directions for the electric field. 
These correspond to the two independent polarizations of electromagnetic waves.

Consider an enclosure whose walls are maintained at fixed
temperature $T$. What is the nature of the 
steady-state electromagnetic radiation inside
the enclosure? Suppose that the enclosure is a parallelepiped with
sides of lengths $L_x$, $L_y$, and $L_z$. Alternatively, suppose that the
radiation field inside the enclosure is periodic in the $x$-, $y$-, and $z$-directions,
with periodicity lengths $L_x$, $L_y$, and $L_z$, respectively. As long as the
smallest of these lengths, $L$, say, is much greater than the longest wavelength
of interest in the problem, $\lambda = 2\pi /k$, then these assumptions should not
significantly 
affect the nature of the radiation inside the enclosure. We find, just as in
our earlier discussion of sound waves (see Sect.~\ref{s7sound}), that the periodicity constraints ensure that
there are only a discrete set of allowed wave-vectors ({\em i.e.}, a discrete
set of allowed modes of oscillation of the electromagnetic field inside the
enclosure).
 Let $\rho({\bf k})\,
d^3{\bf k}$ be the number of allowed modes {\em per unit volume}
with wave-vectors in the range  ${\bf k}$ to ${\bf k} + d{\bf k}$. We know,
by analogy  with
Eq.~(\ref{e7.150}), that
\begin{equation}
\rho({\bf k})\,d^3{\bf k} = \frac{d^3{\bf k}}{(2\pi)^3}.
\end{equation}
The number of modes per unit volume for which the {\em magnitude}\/ of
the wave-vector lies in the range $k$ to $k+dk$ is just the density of modes,
$\rho({\bf k})$, 
multiplied by the ``volume'' in ${\bf k}$-space of the spherical shell
lying between radii $k$ and $k+dk$. Thus,
\begin{equation}
\rho_k(k)\,dk = \frac{4\pi \,k^2\,dk}{(2\pi)^3} = \frac{k^2}{2\pi^2}\, dk.
\end{equation}
Finally, the number of  modes per unit volume whose frequencies lie between
$\omega$ and $\omega + d \omega$ is, by Eq.~(\ref{e8.69}),
\begin{equation}
\sigma (\omega)\,d\omega = 2 \,\frac{\omega^2}{2\pi^2 \,c^3}\,d\omega.
\end{equation}
Here, the additional 
factor 2  is to take account of the two independent polarizations
of the electromagnetic field for a given wave-vector ${\bf k}$. 


Let us consider the situation classically. By analogy with sound waves, we can treat
each allowable mode of oscillation of the electromagnetic field as an independent
harmonic oscillator. According to the equipartition theorem
(see Sect.~\ref{s7eq}), each mode possesses a
mean energy $k\,T$ in thermal equilibrium at temperature $T$. In fact, $(1/2)\,k\,T$
resides with the oscillating electric field, and another $(1/2)\,k\,T$ with
the oscillating magnetic field.
Thus,
the classical {\em energy density} of electromagnetic radiation ({\em i.e.}, the
energy per unit volume associated with modes whose frequencies lie in the
range $\omega$ to $\omega + d\omega$) is
\begin{equation}
\overline{u}(\omega)\,d\omega = k\,T \,\sigma(\omega) \,d\omega = \frac{k\,T}
{\pi^2 \,c^3}
\,\omega^2\,d\omega.
\end{equation}
This result is known as the {\em Rayleigh-Jeans radiation law}, after Lord Rayleigh
and James Jeans who first proposed it in the late nineteenth century.


According to Debye theory (see Sect.~\ref{s7sound}),
  the energy density of sound waves in a solid is 
analogous to the Rayleigh-Jeans law, with one very important difference. 
In Debye theory
there is a cut-off frequency (the Debye frequency) above which 
no modes exist. This cut-off comes about because of the discrete nature of
solids ({\em i.e.}, because solids 
 are made up of atoms instead of being continuous).
It is, of course, 
 impossible to have sound waves whose wavelengths are much less than
the inter-atomic spacing. 
On the other hand, electromagnetic waves propagate through a vacuum,
which possesses no discrete structure. It follows that
there is no cut-off frequency for electromagnetic waves, and so the Rayleigh-Jeans
law holds for all frequencies. This immediately poses a severe problem. The total
classical energy density of  electromagnetic radiation is given by
\begin{equation}
U = \int_0^{\infty} \overline{u}(\omega)
\,d\omega = \frac{k\,T}{\pi^2 \,c^3} \int_0^{\infty}
\omega^2\,d\omega.
\end{equation}
This  is an integral which obviously does not converge. Thus, according to classical
physics, the total energy density of electromagnetic radiation inside  an enclosed cavity
is infinite! This is clearly an absurd result, and was recognized as such
in the latter half of the nineteenth century. In fact, this  prediction
is known as the {\em ultra-violet catastrophe}, because the Rayleigh-Jeans
law usually starts to diverge badly from experimental
observations (by over-estimating the amount of radiation) in the 
ultra-violet region of the spectrum.

So, how do we obtain a sensible answer? Well, as usual, quantum mechanics comes
to our rescue. According to quantum mechanics, 
each allowable mode of oscillation of the
electromagnetic field corresponds to a {\em photon state} with energy and
momentum
\begin{eqnarray}
\epsilon &= &\hbar\,\omega,\\[0,5ex]
{\bf p} &=& \hbar\,{\bf k},
\end{eqnarray}
respectively.
Incidentally, it follows from Eq.~(\ref{e8.69}) that
\begin{equation}
\epsilon = p\,c,
\end{equation}
which implies that photons are massless particles which move at the speed of
light.
According to the Planck distribution (\ref{e8zzz}), the mean number of
photons occupying a photon state of frequency $\omega$ is
\begin{equation}
n(\omega) = \frac{1}{{\rm e}^{\,\beta\,\hbar\,\omega}-1}.
\end{equation}
Hence, the mean energy of such a state is given by
\begin{equation}
\bar{\epsilon}(\omega) = \hbar\,\omega\,\,n(\omega) = \frac{\hbar\,\omega}{{\rm e}^{\,\beta\,\hbar\,\omega}-1}.
\end{equation}
Note that low frequency states ({\em i.e.}, $\hbar\,\omega\ll k\,T$) behave
classically: {\em i.e.},
\begin{equation}
\bar{\epsilon} \simeq k\,T.
\end{equation}
On the other hand, high frequency states ({\em i.e.}, $\hbar\,\omega\gg k\,T$)
are completely ``frozen out'': {\em  i.e.},
\begin{equation}
\bar{\epsilon} \ll k\,T.
\end{equation}
The reason for this is simply that it is very difficult for a thermal
fluctuation to create a photon with an energy greatly in excess of $k\,T$,
since $k\,T$ is the characteristic energy associated with such  fluctuations.

According to the above discussion, the true energy density of
electromagnetic radiation inside an enclosed cavity  is written
\begin{equation}
\bar{u}\,d\omega = \epsilon(\omega)\,\sigma(\omega)\,d\omega,
\end{equation}
giving
\begin{equation}
\overline{u}(\omega)\,d\omega = \frac{\hbar}{\pi^2 \,c^3} \frac{\omega^3\,d\omega}
{\exp(\beta\,\hbar\,\omega) - 1}.
\end{equation}
This is  famous
result is known as the  {\em Planck radiation law}. 
The Planck law approximates to the
classical Rayleigh-Jeans law for $\hbar\,\omega \ll k\,T$, peaks at about
$\hbar\,\omega\simeq 3\, k\,T$, and falls off exponentially for $\hbar\,\omega \gg 
k\,T$.
The exponential fall off at high frequencies ensures that the total energy density
remains finite.

\section{Black-Body Radiation}
Suppose that we were to make a small hole in the wall of our enclosure,
and observe the emitted radiation. A small hole is the best approximation in
Physics to a {\em black-body}, which is defined as an object which absorbs, and,
therefore, emits, radiation perfectly at all wavelengths. 
What is the  power radiated by the hole? Well, the power density inside the enclosure
can be written
\begin{equation}
\overline{u}(\omega)\,d\omega = \hbar\,\omega\,\, n(\omega)\, d\omega,
\end{equation}
where $n(\omega)$ is the mean 
number of photons per unit volume whose frequencies lie
in the range $\omega$ to $\omega + d\omega$. The radiation field inside the
enclosure is isotropic (we are assuming that the hole is sufficiently small that
it does not distort the field). It follows that the mean number of photons
per unit volume 
whose frequencies lie in the specified range, and
whose directions of propagation make an angle in the range
$\theta$ to $\theta + d\theta$ with the  normal to the hole, is 
\begin{equation}
n(\omega, \theta)\,d\omega\,d\theta = 
\frac{1}{2}\,n(\omega)\,d\omega\,\sin\theta\,d\theta,
\end{equation}
where $\sin\theta$ is proportional to the solid angle in the specified 
range of directions,
and
\begin{equation}
\int_0^\pi n(\omega, \theta)\,d\omega\,d\theta = n(\omega)
\,d\omega.
\end{equation}
Photons travel at the velocity of light, so the power per unit area escaping from
the hole in the frequency range $\omega$ to $\omega+d\omega$  is
\begin{equation}
P(\omega)\, d\omega =\int_0^{\pi/2} c\,\cos\theta\,\,\hbar\,\omega\,\,n(\omega, \theta)
\,d\omega\,d\theta,
\end{equation}
where $c\,\cos\theta$ is the component of the photon velocity in the direction
of the hole.
This gives
\begin{equation}
P(\omega)\, d\omega = c \,\,\overline{u}(\omega)\, d\omega\,\frac{1}{2}\!\int_0^{\pi/2}
\cos\theta\,\sin\theta\,d\theta = \frac{c}{4} \,\overline{u}(\omega)\,d\omega,
\end{equation}
so
\begin{equation}
P(\omega)\,d\omega = \frac{\hbar}{4\pi^2\, c^2} \frac{\omega^3\,d\omega}
{\exp(\beta\,\hbar\,\omega)-1}
\end{equation}
is the power per unit area radiated by a black-body in the frequency range
$\omega$ to $\omega + d\omega$. 

A black-body is very much an idealization. The power
spectra of real radiating bodies
can deviate quite substantially from black-body spectra. Nevertheless, we
can make some useful predictions using this model. The black-body power spectrum
peaks when $\hbar\,\omega\,\simeq 3 \,k\,T$. This means that the peak radiation
frequency scales {\rm linearly}\/ with the temperature of the body. In other words,
hot bodies tend to radiate  at higher frequencies than cold bodies. This
result (in particular, the linear scaling) is known as {\em Wien's displacement
law}. It allows us to estimate the surface temperatures of stars from their
colours (surprisingly enough, stars are fairly good black-bodies). Table~\ref{t8} shows
some stellar temperatures determined by this method (in fact,
the whole emission spectrum is fitted to a black-body spectrum). 
It can be seen that the
apparent colours (which correspond quite well to the colours of the peak radiation)
scan the whole visible spectrum, from red to blue, as the stellar surface temperatures
gradually rise.
\begin{table}[h]\centering
\begin{tabular}{llccl}\hline
Name & Constellation & Spectral Type & Surf.\ Temp.\ ($^\circ$~K)& Colour\\\hline
Antares & Scorpio & M & 3300 & Very Red\\
Aldebaran & Taurus & K & 3800 & Reddish\\
Sun&   & G& 5770 & Yellow\\
Procyon & Canis Minor &F& 6570 & Yellowish\\
Sirius & Canis Major &A&  9250 & White\\
Rigel & Orion & B &11,200 & Bluish White\\
\end{tabular}
\caption{\em Physical properties of some well-known stars}\label{t8}
\end{table}

Probably the most famous black-body spectrum is cosmological in origin. Just after
the ``big bang'' the Universe was essentially a ``fireball,'' with the energy associated with
radiation completely dominating that associated with
matter. The early Universe was also pretty
well described by equilibrium statistical thermodynamics,
which means that the radiation had a black-body spectrum. As the Universe expanded,
the radiation was gradually Doppler shifted to ever larger wavelengths (in other
words, the radiation did work against the expansion of the Universe, and, thereby,
lost energy), but its spectrum remained invariant. Nowadays, this primordial
radiation is detectable as a faint {\em microwave background}\/ which pervades the
whole universe. The microwave background was discovered accidentally by Penzias
and Wilson in 1961. Until recently, it was difficult to measure the full
spectrum with any degree of
precision, because of strong microwave absorption and scattering by
the Earth's atmosphere. However, all of this changed  when the COBE satellite
was launched in 1989. It took precisely nine minutes to measure the perfect
black-body spectrum reproduced in Fig.~\ref{fbb}.
 This data can be fitted to a black-body
curve of characteristic
temperature $2.735~^\circ$~K. In a very real sense, this can be regarded
as the ``temperature of the Universe.''

\begin{figure}
\epsfysize=3in
\centerline{\epsffile{Chapter08/cobe.eps}}
\caption{\em Cosmic background radiation spectrum measured by the
Far Infrared Absolute Spectrometer (FIRAS) aboard the Cosmic Background
Explorer satellite (COBE).}\label{fbb}
\end{figure}

\section{Stefan-Boltzmann Law}
The total power radiated per unit area by a black-body at {\em all}
frequencies is given by
\begin{equation}
P_{\rm tot}(T) = \int_0^\infty P(\omega) \, d\omega
= \frac{\hbar}{4\pi^2\, c^2} \int_0^\infty \frac{\omega^3\,d\omega}
{\exp(\hbar\,\omega/k\,T) - 1},
\end{equation}
or
\begin{equation}
P_{\rm tot}(T) =  \frac{k^4\, T^4 }{4\pi^2\, c^2\,\hbar^3} \int_0^\infty
\frac{\eta^3\,d\eta}{\exp\eta -1},
\end{equation}
where $\eta = \hbar\,\omega/k \,T$. The above integral can easily be looked up in
standard mathematical tables. In fact,
\begin{equation}
\int_0^\infty
\frac{\eta^3\,d\eta}{\exp\eta -1} = \frac{\pi^4}{15}.
\end{equation}
Thus, the total power radiated per unit area by a black-body is
\begin{equation}
P_{\rm tot}(T) = \frac{\pi^2}{60} \frac{k^4}{c^2\, \hbar^3} \,T^4 = \sigma \,T^4.
\end{equation}
This $T^4$ dependence of the radiated power is called the {\em Stefan-Boltzmann law},
after Josef Stefan, who first obtained it
experimentally, and Ludwig Boltzmann, who first derived it theoretically.
The parameter
\begin{equation}
\sigma = \frac{\pi^2}{60} \frac{k^4}{c^2\, \hbar^3} = 5.67\times 10^{-8}\,\,{\rm W}
\,{\rm m}^{-2} \,{\rm K}^{-4},
\end{equation}
is called the {\em Stefan-Boltzmann constant}.

We can use the Stefan-Boltzmann law to estimate the temperature of the Earth
from first principles. The Sun is a ball of glowing gas of radius $R_\odot\simeq
7\times 10^5$ km and surface temperature $T_\odot\simeq 5770^\circ$~K. Its
luminosity is 
\begin{equation}
L_\odot = 4\pi\, R_\odot^{~2} \, \sigma\, T_\odot^{~4},
\end{equation}
according to  the Stefan-Boltzmann law. The Earth is a globe of radius
$R_\oplus\sim 6000$~km located an average distance $r_\oplus\simeq 1.5\times 10^8$~km
from the Sun. The Earth intercepts an amount of energy
\begin{equation}
P_\oplus=L_\odot\,\frac{ \pi \,R_\oplus^{~2}/r_\oplus^{~2}}{4\pi}
\end{equation}
per second from the Sun's radiative output: {\em i.e.}, the power output of the Sun 
reduced by the ratio of the solid angle subtended by the Earth at the Sun to
the total solid angle $4\pi$. The Earth absorbs this energy, and then
re-radiates it at longer wavelengths. The luminosity of the Earth is
\begin{equation}
L_\oplus = 4\pi\, R_\oplus^{~2} \, \sigma\, T_\oplus^{~4},
\end{equation}
according to the Stefan-Boltzmann law,
where $T_\oplus$ is the average temperature of the Earth's surface. 
Here, we are  ignoring
any surface
temperature variations between polar and equatorial regions, or between day
and night. In steady-state, the luminosity of the Earth must balance the radiative
power input from the Sun,
so equating $L_\oplus$ and $P_\oplus$ we arrive at
\begin{equation}
T_\oplus = \left(\frac{R_\odot}{2\,r_\oplus}\right)^{1/2} T_\odot.
\end{equation}
Remarkably, the ratio of the Earth's surface temperature to that of the Sun depends
only on the Earth-Sun distance and the solar radius. The above expression
yields $T_\oplus\sim 279^\circ$~K or $6^\circ$~C (or $43^\circ$~F). This is slightly on the cold side, by a few
degrees, because of the
greenhouse action of the Earth's atmosphere,
which was neglected in our
calculation. Nevertheless, it is quite encouraging that such a crude calculation
comes so close to the correct answer.

\section{Conduction Electrons in a Metal}
The conduction electrons in a metal are non-localized ({\em i.e.}, they are
not tied to any particular atoms). In conventional metals, each atom contributes
a single such electron. To a first approximation, it is possible
to neglect the mutual interaction of the conduction electrons, since this
interaction
is largely shielded out by the stationary atoms. The conduction electrons
can, therefore, be treated as an {\em ideal gas}. However, the concentration of
such  electrons in a metal {\em far exceeds}\/ the concentration of
particles in a conventional gas. It is, therefore, not surprising that 
conduction electrons cannot normally be analyzed using classical statistics:
in fact,  they are subject to Fermi-Dirac statistics
(since electrons are fermions).

Recall, from Sect.~\ref{s8fd}, that the mean number of particles occupying
state $s$ (energy $\epsilon_s$) is given by
\begin{equation}
\bar{n}_s = \frac{1}{{\rm e}^{\,\beta\,(\epsilon_s-\mu)} + 1},
\end{equation}
according to the Fermi-Dirac distribution. Here,
\begin{equation}
\mu \equiv -k\,T\,\,\alpha
\end{equation}
is termed the {\em Fermi energy}\/ of the system. This energy is
determined by the condition that
\begin{equation}
\sum_r \bar{n}_r = \sum_r\frac{1}{{\rm e}^{\,\beta\,(\epsilon_r-\mu)} + 1}
=N,
\end{equation}
where $N$ is the total number of particles contained in the volume $V$. 
It is clear, from the above equation, that the Fermi energy $\mu$  is generally
a function of the temperature $T$.

Let us investigate the behaviour of the {\em Fermi function}
\begin{equation}
F(\epsilon) = \frac{1}{{\rm e}^{\,\beta\,(\epsilon-\mu)} + 1}
\end{equation}
as $\epsilon$ varies. Here, the energy is measured from
its lowest possible value $\epsilon=0$. If the Fermi energy $\mu$ is
such that $\beta\,\mu\ll 1$ then $\beta\,(\epsilon-\mu)\gg 1$,
and $F$ reduces to the Maxwell-Boltzmann distribution. However,
for the case of conduction electrons in a metal we are interested in
the opposite limit, where 
\begin{equation}
\beta\,\mu \equiv \frac{\mu}{k\,T} \gg 1.
\end{equation}
In this limit, if $\epsilon\ll \mu$ then $\beta\,(\epsilon-\mu)\ll 1$,
so that $F(\epsilon)=1$. On the other hand, if $\epsilon\gg \mu$ then
$\beta\,(\epsilon-\mu)\gg 1$, so that $F(\epsilon)= 
\exp[-\beta\,(\epsilon-\mu)]$ falls off exponentially with increasing
$\epsilon$, just like a classical Boltzmann distribution. Note that
$F=1/2$ when $\epsilon=\mu$. The transition region in which $F$ goes from 
a value close to unity to a value close to zero corresponds to an
energy interval of order $k\,T$, centred on $\epsilon=\mu$. This is
illustrated in Fig.~\ref{ffermi}.

\begin{figure}
\epsfysize=3in
\centerline{\epsffile{Chapter08/fermi.eps}}
\caption{\em The Fermi function.}\label{ffermi}
\end{figure}

In the limit as $T\rightarrow 0$, the transition region becomes
infinitesimally narrow. In this case, $F=1$ for $\epsilon<\mu$ and
$F=0$ for $\epsilon>\mu$, as illustrated in Fig.~\ref{ffermi}.
This is an obvious result, since when $T=0$ the conduction
electrons attain their lowest energy, or  {\em ground-state}, configuration.
Since the Pauli exclusion principle requires that there be no
more than one electron per single-particle quantum state, the lowest
energy configuration is obtained by piling
electrons into the lowest available unoccupied states until all of
the electrons are used up. Thus, the last electron added to the
pile has quite a considerable energy, $\epsilon= \mu$, since all of the
lower energy states are already occupied. Clearly, the exclusion principle
implies that a Fermi-Dirac gas possesses a large mean energy, even at absolute
zero.

Let us calculate the Fermi energy $\mu=\mu_0$ of a Fermi-Dirac
gas at $T=0$. The energy of each particle is related to its
momentum\/ ${\bf p} =\hbar\,{\bf k}$ via
\begin{equation}
\epsilon = \frac{p^2}{2\,m} =\frac{\hbar^2\,k^2}{2\,m},
\end{equation}
where ${\bf k}$ is the de~Broglie wave-vector. At $T=0$ all quantum states
whose energy is less than the Fermi energy $\mu_0$ are filled. The
Fermi energy corresponds to a {\em Fermi momentum}\/ $p_F=\hbar\,k_F$ which
is such that
\begin{equation}\label{e8fe}
\mu_0 = \frac{p_F^{\,2}}{2\,m} = \frac{\hbar^2\,k_F^{\,2}}{2\,m}.
\end{equation}
Thus, at $T=0$ all quantum states with $k<k_F$ are filled, and all
those with $k>k_F$ are empty.

Now, we know, by analogy with Eq.~(\ref{e7.150}), that there are $(2\pi)^{-3}\,V$
allowable translational states per unit volume of ${\bf k}$-space. The volume of
the sphere of radius $k_F$ in ${\bf k}$-space is $(4/3)\,\pi\,k_F^{\,3}$. It
follows that the {\em Fermi sphere}\/ of radius $k_F$ contains 
$(4/3)\,\pi\,k_F^{\,3}\,(2\pi)^{-3}\,V$ translational states. The number of
quantum states inside the sphere is {\em twice}\/ this, because electrons
possess two possible spin states for every possible translational state. Since the
total number of occupied states ({\em i.e.}, the total number of quantum
states inside the Fermi sphere) must equal the total number of particles
in the gas, it follows that
\begin{equation}
2\,\frac{V}{(2\pi)^3}\left(\frac{4}{3}\,\pi\,k_F^{\,3}\right) = N.
\end{equation}

The above expression can be rearranged to
give
\begin{equation}
k_F = \left(3\,\pi^2\,\frac{N}{V}\right)^{1/3}.
\end{equation}
Hence,
\begin{equation}
\lambda_F \equiv \frac{2\pi}{k_F} =\frac{2\pi}{(3\pi^2)^{1/3}}
\left(\frac{V}{N}\right)^{1/3},
\end{equation}
which implies that the de~Broglie wavelength $\lambda_F$
corresponding to the Fermi energy is of order the mean separation
between particles $(V/N)^{1/3}$. All quantum states with de~Broglie
wavelengths $\lambda\equiv 2\pi/k > \lambda_F$ are occupied
at $T=0$, whereas all those with $\lambda<\lambda_F$ are empty.

According to Eq.~(\ref{e8fe}), the Fermi energy at $T=0$ takes the form
\begin{equation}
\mu_0 = \frac{\hbar^2}{2\,m}\left(3\,\pi^2\,\frac{N}{V}\right)^{2/3}.
\end{equation}
It is easily demonstrated that $\mu_0\gg k\,T$ for conventional metals
at room temperature.

The majority of the conduction electrons in a metal occupy a
band of completely filled states with energies far below the Fermi
energy. In many cases, such electrons have very little effect
on the macroscopic properties of the metal. Consider, for example, the
contribution of the conduction electrons to the specific heat of the metal.
The heat capacity $C_V$ at constant volume of these electrons can
be calculated from a knowledge of their mean energy $\bar{E}(T)$ as a
function of $T$: {\em i.e.},
\begin{equation}
C_V = \left(\frac{\partial \bar{E}}{\partial T}\right)_V.
\end{equation} 
If the electrons obeyed classical Maxwell-Boltzmann statistics, so that
$F\propto \exp(-\beta\,\epsilon)$ for {\em all}\/ electrons, then the
equipartition theorem would give
\begin{eqnarray}
\bar{E} &=& \frac{3}{2}\,N\,k\,T,\\[0.5ex]
C_V &=& \frac{3}{2}\,N\,k.\label{e8hc}
\end{eqnarray}
However, the actual situation, in which $F$ has the form shown in Fig.~\ref{ffermi},
is very different. A small change in $T$ does not affect the
mean  energies of
the majority of the electrons, with $\epsilon\ll \mu$, since these electrons
lie in states which are completely filled, and remain so when the temperature
is changed. It follows that these electrons contribute nothing whatsoever to
the heat capacity. On the other hand, the relatively small number of
electrons $N_{\rm eff}$ in the energy range of order $k\,T$, centred on the
Fermi energy, in which $F$ is significantly different from 0 and 1,
do contribute to the specific heat. In the tail end of this region
$F\propto \exp(-\beta\,\epsilon)$, so the
distribution reverts to a Maxwell-Boltzmann distribution.
Hence, from Eq.~(\ref{e8hc}),
we expect each electron in this region to contribute roughly an amount
$(3/2)\,k$ to the heat capacity. Hence, the heat capacity
can be written
\begin{equation}
C_V\simeq \frac{3}{2}\,N_{\rm eff}\,k.
\end{equation}
However, since only a fraction $k\,T/\mu$ of the total conduction
electrons lie in the tail region of the Fermi-Dirac distribution, we
expect
\begin{equation}
N_{\rm eff} \simeq \frac{k\,T}{\mu}\,N.
\end{equation}
It follows that
\begin{equation}
C_V \simeq \frac{3}{2}\,N\,k\,\frac{k\,T}{\mu}.\label{e8res}
\end{equation}

Since $k\,T\ll \mu$ in conventional metals, the molar specific heat of
the conduction electrons is clearly very much less than the classical value
$(3/2)\,R$. This accounts for the fact that
the molar specific heat capacities of metals at room temperature are
about the same as those of insulators. Before the advent of
quantum mechanics, the classical theory predicted incorrectly that the
presence of conduction electrons should raise the heat capacities of
metals by 50 percent [{\em i.e.}, $(3/2)\,R$] compared to those of insulators.

Note that the specific heat (\ref{e8res}) is not temperature independent.
In fact, using the superscript $e$ to denote the {\em electronic}
specific heat, the molar specific heat can be written
\begin{equation}
c_V^{(e)} = \gamma\,T,
\end{equation}
where $\gamma$ is a (positive) constant of proportionality.
At room temperature $c_V^{(e)}$ is completely masked by the much
larger specific heat $c_V^{(L)}$ due to lattice vibrations. However, at
very low temperatures $c_V^{(L)}=A\,T^3$, where $A$ is a (positive)
constant of proportionality (see Sect.~\ref{s7sound}). Clearly, 
at low temperatures $c_V^{(L)}=A\,T^3$ approaches zero far more rapidly that
the electronic specific heat, as $T$ is reduced. Hence, it should be possible
to measure the electronic contribution to the molar specific
heat at low temperatures. 

The total molar specific heat of a metal at low temperatures takes the
form
\begin{equation}
c_V = c_V^{(e)} + c_V^{(L)} = \gamma\,T + A\,T^3.\label{e8td}
\end{equation}
Hence,
\begin{equation}
\frac{c_V}{T} = \gamma + A\,T^2.
\end{equation}
It follows that a plot of $c_V/T$ versus $T^2$ should yield a
{\em straight line}\/ whose intercept on the vertical axis gives the
coefficient $\gamma$. Figure~\ref{fkcv} shows such a plot.
The fact that a good straight line is obtained verifies that the temperature
dependence of the heat capacity predicted by Eq.~(\ref{e8td}) is  indeed
correct.

\begin{figure}
\epsfysize=2in
\centerline{\epsffile{Chapter08/kcv.eps}}
\caption{\em The low temperature heat capacity of potassium, plotted as $C_V/T$ versus
$T^2$. From C.~Kittel, and H.~Kroemer, Themal physics (W.H.~Freeman \& co., New York NY, 1980).}\label{fkcv}
\end{figure}

\section{White-Dwarf Stars}
A main-sequence hydrogen-burning star, such as the Sun, is maintained
in equilibrium via the balance of the gravitational attraction tending
to  make it collapse, and the thermal pressure tending to make it expand. 
Of course, the
thermal energy of the star is generated by nuclear reactions occurring deep inside
its core. Eventually, however, the star will run out of burnable fuel, and, therefore,
start to collapse, as it radiates away its remaining thermal energy.
What is the ultimate fate of such a star?

A burnt-out star is basically a gas of electrons and ions. As the
star collapses, its density increases, so the mean separation between its
constituent particles decreases. Eventually, the mean separation becomes
of order the de~Broglie wavelength of the electrons, and the electron
gas becomes {\em degenerate}. Note, that the de~Broglie wavelength of the
ions is  much smaller than that of the electrons, so the ion gas remains
non-degenerate. Now, even at
zero temperature, a degenerate electron gas exerts a substantial pressure,
because the Pauli exclusion principle prevents the mean electron separation
from becoming significantly  smaller than the typical 
de~Broglie wavelength (see the
previous section). Thus, it is possible for a burnt-out star to maintain
itself against complete collapse under gravity via the {\em degeneracy pressure}
of its constituent electrons. Such stars are termed {\em white-dwarfs}. 
Let us investigate the physics of white-dwarfs in more detail.

The total energy of a white-dwarf star can be written
\begin{equation}
E = K + U,\label{e8wd1}
\end{equation}
where $K$ is the total kinetic energy of the degenerate electrons (the kinetic
energy of the ion is negligible) and $U$ is the gravitational potential
energy. Let us assume, for the sake of simplicity, that the density of the
star is {\em uniform}. In this case, the gravitational potential
energy takes the form
\begin{equation}
U = -\frac{3}{5}\frac{G\,M^2}{R},\label{e8wd2}
\end{equation}
where $G$ is the gravitational constant, $M$ is the stellar mass, and $R$ is
the stellar radius.

Let us assume that the electron gas is highly degenerate, which is
equivalent to taking the limit $T\rightarrow 0$. In this case, we know,
from the previous section, that the Fermi momentum can be written
\begin{equation}
p_F = {\mit\Lambda}\left(\frac{N}{V}\right)^{1/3},
\end{equation}
where
\begin{equation}
{\mit\Lambda} = (3\pi^2)^{1/3}\,\hbar.\label{e8wd3}
\end{equation}
Here,
\begin{equation}
V = \frac{4\pi}{3}\,R^3\label{e8wd4}
\end{equation}
is the stellar volume, and $N$ is the total number of electrons contained
in the star. Furthermore, the number of
electron states contained in an annular radius of ${\bf p}$-space
lying between radii $p$ and $p+dp$ is
\begin{equation}
dN = \frac{3\,V}{{\mit\Lambda}^3}\, p^2\,dp.
\end{equation}
Hence, the total kinetic energy of the electron gas can be written
\begin{equation}
K = \frac{3\,V}{{\mit\Lambda}^3}\int_0^{p_F} \frac{p^2}{2\,m}\,p^2\,dp =
\frac{3}{5}\,\frac{V}{{\mit\Lambda}^3}\,\frac{p_F^{\,5}}{2\,m},\label{e8wdd}
\end{equation}
where $m$ is the electron mass. It follows that
\begin{equation}
K = \frac{3}{5}\,N\,\frac{{\mit\Lambda}^2}{2\,m}\left(\frac{N}{V}\right)^{2/3}.\label{e8wd5}
\end{equation}

The interior of a white-dwarf star is composed of atoms like
$C^{12}$ and $O^{16}$ which contain equal numbers of protons, neutrons, and
electrons. Thus,
\begin{equation}
M = 2\,N\,m_p,\label{e8wd6}
\end{equation}
where $m_p$ is the proton mass. 

Equations (\ref{e8wd1}), (\ref{e8wd2}), (\ref{e8wd3}), (\ref{e8wd4}),
(\ref{e8wd5}), and (\ref{e8wd6}) can be combined to give
\begin{equation}
E = \frac{A}{R^2} - \frac{B}{R},
\end{equation}
where
\begin{eqnarray}
A &=& \frac{3}{20}\left(\frac{9\pi}{8}\right)^{2/3}\frac{\hbar^2}{m}
\left(\frac{M}{m_p}\right)^{5/3},\\[0.5ex]
B&=& \frac{3}{5}\,G\,M^2.
\end{eqnarray}
The equilibrium radius of the star $R_\ast$ is that which
{\em minimizes} the total energy $E$. In fact,
it is easily demonstrated that
\begin{equation}
R_\ast = \frac{2\,A}{B},
\end{equation}
which yields
\begin{equation}
R_\ast = \frac{(9\pi)^{2/3}}{8}\,\frac{\hbar^2}{m}\,\frac{1}{G\,m_p^{\,5/3}\,
M^{1/3}}.
\end{equation}
The above formula can also be written
\begin{equation}
\frac{R_\ast}{R_\odot}= 0.010\left(\frac{M_\odot}{M}\right)^{1/3},\label{e8wd9}
\end{equation}
where $R_\odot= 7\times 10^5\,{\rm km}$ is the solar radius, and
$M_\odot = 2\times 10^{30}\,{\rm kg}$ is the solar mass. It follows that
the radius of a typical solar mass white-dwarf is about 7000\,km: 
{\em i.e.}, about the same as the radius of the Earth. The first
white-dwarf to be discovered (in 1862) was the companion of Sirius. Nowadays,
thousands of white-dwarfs have been observed, all with properties similar
to those described above.

\section{Chandrasekhar Limit}
One curious feature of white-dwarf stars is that their radius decreases
as their mass increases [see Eq.~(\ref{e8wd9})]. It follows, from Eq.~(\ref{e8wd5}),
that the mean energy of the degenerate electrons inside the
star increases strongly as the stellar
mass increases: in fact, $K\propto M^{4/3}$. Hence, if 
$M$ becomes sufficiently large  the electrons become {\em relativistic}, and
the above analysis needs to be modified. Strictly speaking, the non-relativistic
analysis described in the previous section
 is only valid in the low mass limit $M\ll M_\odot$. 
Let us, for the sake of simplicity, consider the ultra-relativistic
limit in which $p\gg m\,c$. 

The total electron energy (including the rest mass energy) can be
written
\begin{equation}
K = \frac{3\,V}{{\mit\Lambda}^3}\int_0^{p_F} (p^2\,c^2
+m^2\,c^4)^{1/2}\,p^2\,dp,
\end{equation}
by analogy with Eq.~(\ref{e8wdd}). Thus,
\begin{equation}
K\simeq \frac{3\,V\,c}{{\mit\Lambda}^3}\int_0^{p_F} \left(p^3
+ \frac{m^2\,c^2}{2}\,p + \cdots\right)dp,
\end{equation}
giving
\begin{equation}
K \simeq \frac{3}{4}\,\frac{V\,c}{{\mit\Lambda}^3}
\left[p_F^{\,4} + m^2\,c^2\,p_F^{\,2}+\cdots \right].
\end{equation}

It follows, from the above, that the total energy of an ultra-relativistic
white-dwarf star can be written
in the
form
\begin{equation}
E \simeq \frac{A-B}{R} + C\,R,
\end{equation}
where
\begin{eqnarray}
A &=& \frac{3}{8}\left(\frac{9\pi}{8}\right)^{1/3} \!\hbar\,c\left(\frac{M}{m_p}
\right)^{4/3},\\[0.5ex]
B&=& \frac{3}{5}\,G\,M^2,\\[0.5ex]
C&=& \frac{3}{4}\,\frac{1}{(9\pi)^{1/3}}\,
\frac{m^2\,c^3}{\hbar} \left(\frac{M}{m_p}\right)^{2/3}.
\end{eqnarray}
As before, the equilibrium radius $R_\ast$ is that which minimizes the
total energy $E$. 
However, in the ultra-relativistic case, a non-zero value of $R_\ast$ only exists
for $A-B>0$. When $A-B<0$ the energy decreases monotonically with decreasing
stellar radius: in other words, the degeneracy pressure 
of the electrons is incapable of halting the collapse of the star under gravity.
The criterion which must be satisfied for a relativistic white-dwarf
star to be maintained against gravity is that
\begin{equation}
\frac{A}{B} > 1.
\end{equation}
This criterion can be re-written
\begin{equation}
M< M_C,
\end{equation}
where
\begin{equation}
M_C = \frac{15}{64}\,(5\pi)^{1/2} \,\frac{(\hbar\,c/G)^{1/2}}{m_p^{\,2}}=
1.72\,M_\odot
\end{equation}
is known as the {\em Chandrasekhar limit}, after  A.~Chandrasekhar
who first derived it in 1931.
A  more realistic calculation, which does not assume constant density,
yields
\begin{equation}
M_C = 1.4\,M_\odot.
\end{equation}
Thus, if the stellar mass exceeds the Chandrasekhar limit then the star in question
cannot become a white-dwarf when its nuclear fuel is exhausted, but, instead,
must continue to 
collapse. What is the ultimate fate of such a star?

\section{Neutron Stars}
At stellar densities which greatly exceed white-dwarf densities, the
extreme pressures cause electrons to combine with 
protons to form neutrons. Thus, any star which collapses to such an extent that 
its
radius becomes significantly less than that characteristic
of a  white-dwarf
is  effectively  transformed into  a gas  of
neutrons. Eventually, the mean separation between the neutrons becomes
comparable with their de~Broglie wavelength. At this point, it is possible
for the degeneracy pressure of the neutrons to halt the collapse of the star.
A star which is maintained against gravity in this manner is called
a {\em neutron star}. 

Neutrons stars can be analyzed in a very similar manner to
white-dwarf stars. In fact, the previous analysis can be simply modified
by letting $m_p\rightarrow m_p/2$ and $m\rightarrow m_p$. 
Thus, 
we conclude that
 non-relativistic neutrons stars satisfy the mass-radius law:
\begin{equation}
\frac{R_\ast}{R_\odot}= 0.000011\left(\frac{M_\odot}{M}\right)^{1/3},
\end{equation}
It follows that the radius of a typical solar mass neutron star is
a mere 10\,km. In 1967 Antony Hewish and 
Jocelyn Bell discovered a class of compact radio sources, called {\em pulsars},
 which emit
extremely regular pulses of radio waves. Pulsars have subsequently
been identified as
rotating neutron stars. To date, many hundreds of these objects have been
observed.

When relativistic effects are taken into account, 
it is found that there is
a critical mass above which a neutron star cannot be maintained against
gravity. 
According to our analysis,
this critical mass, which is known as the {\em Oppenheimer-Volkoff
limit}, is given by
\begin{equation}
M_{OV} = 4\,M_C = 6.9\,M_\odot.
\end{equation}
A more realistic calculation, which 
does not assume constant density, does not
treat the neutrons as point particles, and takes general relativity into account,
gives a somewhat lower value of
\begin{equation}
M_{OV} = \mbox{1.5---2.5}\,M_\odot.
\end{equation}
A star whose mass exceeds the Oppenheimer-Volkoff
limit cannot be maintained against gravity by degeneracy pressure, and
must ultimately collapse to form a {\em black-hole}.